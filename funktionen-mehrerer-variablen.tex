% ========================================= TEMPLATE INFO ========================================
%
% Author:       P4ntomime
% Version:      1.0.0
% Last updated: 2024-02-18
% Brief:        A LaTeX template for summaries. See README.md for more information.
% 
% ================================================================================================
\documentclass[8pt, a4paper, twoside]{extarticle}
% Font size:    8pt
% Paper size:   A4
% style:        twoside (needed, so odd and even pages have different margins)
% orientation:  portrait. (use 'landscape' for landscape orientation)


% ========================================= DOCUMENT INFO =========================================
\def\title{Funktionen mehrerer Variablen}                               % title
\def\shorttitle{FuVar}                                                  % short title (displayed as PDF title)
\def\dozent{Prof. Dr. Bernhard Zgraggen}                                % lecturer
\def\semester{FS 2024}                                                  % semester
\def\author{Laurin Heitzer, Flurin Brechbühler}                         % author(s)
\def\repo{https://github.com/P4ntomime/funktionen-mehrerer-variablen}   % repository link
\def\version{0.1.\today}                                                % version
\def\pagelimit{12}                                                      % page limit -> causes pages after limit to be red


% ================================= PACKAGES, SETUP AND COMMANDS ==================================
% =========================================== PACKAGES ============================================
\usepackage[utf8]{inputenc}         % input encoding: UTF-8
\usepackage[T1]{fontenc}            % font encoding: T1
\usepackage{textcomp}               % additional symbols
\usepackage{times}                  % times new roman font
\usepackage[main=ngerman]{babel}    % set main language to german


\usepackage{multicol}               % provides multicols environment
\usepackage{geometry}               % set page layout


\usepackage{enumitem}               % list customization
\usepackage{outlines}               % easy nested lists
\usepackage{tabularx}               % some nicer tables with X columns
\usepackage{multirow}               % multirow in tables
\usepackage{hhline}                 % double lines in tables


\usepackage{amsmath}                % math symbols
\usepackage{amssymb}                % more math symbols
\usepackage{mathtools}
\usepackage{mhsetup}
\usepackage{txfonts}
\usepackage[squaren]{SIunits}       % SI-units
\usepackage{bm}                     % bold math symbols
\usepackage{trfsigns}               % needed for "Laplace" symbol (Korrespondenz)
\usepackage{mathrsfs}               % needed for Fourier transform "F"


\usepackage{graphicx}               % include graphics
\usepackage{graphbox}               % needed for aligning images in multicol environment
\usepackage{scalerel}               % scale any objects
\usepackage{anyfontsize}            % set any font size
\usepackage{xcolor}                 % needed for colors


\usepackage{tcolorbox}              % colored boxes
\usepackage{contour}                % contour for text (used in custom underline command)
\usepackage[normalem]{ulem}         % custom underline (used in custom underline command)


\usepackage{tikz}                   % needed for TikZ drawings


% \usepackage{listings}               % for nicer code display
% to use nodes inside listing see: https://texample.net/tikz/examples/tikz-listings/


\usepackage{hyperref}               % clickable links
\usepackage{qrcode}                 % QR code generation (also clickable)


\usepackage{ifthen}                 % if-then-else commands
\usepackage{calc}                   % simple arithmetic in LaTeX commands


\usepackage{draftwatermark}         % watermark on pages after a certain limit
\usepackage{fancyhdr}               % custom header and footer
\usepackage[explicit]{titlesec}     % custom section titles


\usepackage{datetime2}              % custom date format for versioning


% ========================================== BASIC SETUP ==========================================

% --------------------------------------- DOCUMENT SETTINGS ---------------------------------------
\hypersetup{hidelinks,
% set pdf metadata
            pdfauthor={\author},
            pdftitle={\shorttitle},
            pdfsubject={\title\ \semester},
            pdfkeywords={Gahn go lerne!!}}

% set style for URLs
\urlstyle{same} % sets url font to the same as the preceeding text

% set page layout
\geometry{left=3mm, 
          right=3mm, 
          top=3mm, 
          bottom=6mm, 
          headheight=0mm, 
          headsep=0mm, 
          footskip=4mm}

\setlength{\columnsep}{1.5mm}       % distance between columns
\setlength{\columnseprule}{0.1pt}   % thickness of column separation line
\setlength{\parindent}{0pt}         % no paragraph indentation

\setcounter{tocdepth}{2}            % only display sections and subsections in toc
% \setcounter{secnumdepth}{0}       % uncomment to disable section numbering

\DeclareMathSizes{8}{8}{6}{4}       % set math font sizes for 8pt document


% --------------------------------------- COLOR DEFINITIONS ---------------------------------------
\definecolor{sectioncolor}{HTML}{8acafc}
\definecolor{subsectioncolor}{HTML}{c3e3fd}
\definecolor{sectextcol}{HTML}{000000}
\definecolor{subsectextcol}{HTML}{000000}

\definecolor{backcolour}{HTML}{f5f5f0} % background color for highlighted text

% TODO: define color palette
% color palette: https://colorkit.co/color-palette-generator/FF8552-9e22bd-404E7C-C32E15-225A28/
\definecolor{green}{HTML}{1af430}
\definecolor{red}{HTML}{f90d09}
\definecolor{blue}{HTML}{093ce5}
\definecolor{orange}{HTML}{f7730e}
\definecolor{violet}{HTML}{a516c9}

% colors for listings (code)
% \definecolor{commentcolour}{HTML}{404E7C}
% \definecolor{keywordcolour}{HTML}{225A28}
% \definecolor{stringcolour}{HTML}{9e22bd}
% \definecolor{numbercolour}{HTML}{808080}


% ----------------------------------- LIST AND TABULAR SETTINGS -----------------------------------
\setlist[enumerate]{label=\bfseries\arabic*.,   % label style bold arabic numerals (1., 2., ...)
                    leftmargin=*}               % remove left margin from enumerate
\setlist[itemize]{leftmargin=1.5em}             % left margin for itemize: 1.5em
\setlist{nosep}                                 % no vertical spacing between list items

\renewcommand{\arraystretch}{1.2}               % stretch table rows


% ----------------------------------------- TIKZ SETTINGS -----------------------------------------
\usetikzlibrary{arrows}
\usetikzlibrary{arrows.meta}
\usetikzlibrary{bending}
\usetikzlibrary{decorations.pathreplacing}
\usetikzlibrary{angles}
\usetikzlibrary{tikzmark}
\usetikzlibrary{petri}
\usetikzlibrary{positioning}
\usetikzlibrary{shapes}
\usetikzlibrary{calc}
\usetikzlibrary{fit}


% ------------------------------------ OTHER PACKAGE SETTINGS -------------------------------------

% define and set new date style for versioning as YYYYMMDD
\DTMnewdatestyle{vnumdate}{%
    \renewcommand{\DTMdisplaydate}[4]{\number##1\DTMtwodigits{##2}\DTMtwodigits{##3}}%
    \renewcommand{\DTMDisplaydate}{\DTMdisplaydate}%
}
\DTMsetdatestyle{vnumdate}


% setup for ulem and contour packages
\renewcommand{\ULdepth}{1.75pt} % set underline depth
\contourlength{0.7pt}           % set contour length


% ====================================== SETUP AND COMMANDS =======================================

% custom underline command for exclusions on lowercase letters such as g, j, p, q, y
\newcommand{\myul}[1]{%
    \uline{\phantom{#1}}%
    \llap{\contour*{white}{#1}}%
}


% setup header and footer
\pagestyle{fancy}
\fancyhf{}                          % clear all header and footer fields
\renewcommand{\headrulewidth}{0pt}  % remove header rule
\renewcommand{\footrulewidth}{0pt}  % remove footer rule
\fancyfoot[C]{\thepage}             % page number in center of footer


% --------------------------------------- TITLE FORMATTING ----------------------------------------

% section formatting
\titleformat{\section}
            % {\fontsize{9}{8}\selectfont\bfseries}
            {\Large\bfseries}
            {}
            {0mm}
            {\tikz{
                \node[fill=sectioncolor,            % fill color:       sectioncolor
                      text=sectextcol,              % text color:       sectextcol
                      text width=\columnwidth-4pt,  % text width:       columnwidth - 2x padding
                      text depth=0pt,               % text depth:       0pt (needed so text stays vertically centered)
                      minimum height=5mm,           % minimum height:   5mm
                      inner sep=2pt,                % inner padding:    2pt
                      align=left]                   % text alignment:   left
                      {\thesection\ #1};}}

\titleformat{name=\section, numberless}
            % {\fontsize{9}{8}\selectfont\bfseries}
            {\Large\bfseries}
            {}
            {0mm}
            {\tikz{
                \node[fill=sectioncolor,            % fill color:       sectioncolor
                      text=sectextcol,              % text color:       sectextcol
                      text width=\columnwidth-4pt,  % text width:       columnwidth - 2x padding
                      text depth=0pt,               % text depth:       0pt (needed so text stays vertically centered)
                      minimum height=5mm,           % minimum height:   5mm
                      inner sep=2pt,                % inner padding:    2pt
                      align=left]                   % text alignment:   left
                      {#1};}}

\titlespacing{\section}
             {0mm}
             {.8ex}
             {.8ex}


% subsection formatting
\titleformat{\subsection}
            {\large\bfseries}
            {}
            {0mm}
            {\phantomsection\tikz{
                \node[fill=subsectioncolor,         % fill color:       subsectioncolor 
                      text=subsectextcol,           % text color:       subsectextcol 
                      text width=\columnwidth-4pt,  % text width:       columnwidth - 2x padding 
                      text depth=0pt,               % text depth:       0pt (needed so text stays vertically centered)
                      minimum height=5mm,           % minimum height:   5mm 
                      inner sep=2pt,                % inner padding:    2pt 
                      align=left]                   % text alignment:   left
                      {\thesubsection\ #1};}}

\titleformat{name=\subsection, numberless}
            {\large\bfseries}
            {}
            {0mm}
            {\phantomsection\tikz{
                \node[fill=subsectioncolor,         % fill color:       subsectioncolor 
                      text=subsectextcol,           % text color:       subsectextcol 
                      text width=\columnwidth-4pt,  % text width:       columnwidth - 2x padding 
                      minimum height=5mm,           % minimum height:   5mm 
                      inner sep=2pt,                % inner padding:    2pt 
                      align=left]                   % text alignment:   left
                      {#1};}}

\titlespacing{\subsection}
             {0mm}
             {.8ex}
             {.8ex}


% subsubsection formatting
\titleformat{\subsubsection}
            {\fontsize{10}{9}\selectfont\bfseries}
            % {\fontsize{9}{8}\selectfont\bfseries}
            % {\normalsize\bfseries}
            {\hspace{2pt}\myul{\thesubsubsection\ }}
            {0mm}
            {\phantomsection{}\myul{#1}}

\titlespacing{\subsubsection}
             {0mm}
             {.8ex}
             {.8ex}


% custom command for examples
\newcommand{\example}[1]{\subsubsection*{Beispiel: #1}}


% ----------------------------------- CUSTOM TABULAR SPECIFIERS -----------------------------------

% centered fixed width column type
\newcolumntype{P}[1]{>{\centering\arraybackslash}p{#1}}

 % centered variable width column type
\newcolumntype{C}{>{\centering\arraybackslash}X}

% centered math column type
\newcolumntype{M}{>{$}c<{$}}

% right aligned math column type
\newcolumntype{O}{>{$}r<{$}}


% inline tikz node for later referencing
\newcommand{\tikznode}[2]{% from https://tex.stackexchange.com/a/402466/121799
	\ifmmode%
	\tikz[remember picture,baseline= (#1.base),inner sep=0pt] \node(#1){$#2$};
	\else
	\tikz[remember picture,baseline= (#1.base),inner sep=0pt] \node(#1){#2};
	\fi}


% custom inline tcolorbox
\newtcbox{\mybox}
            [1]
            [backcolour]
            {on line,
            arc=0pt,
            outer arc=0pt,
            colback=#1,
            colframe=#1,
            boxsep=0pt,
            left=1pt,
            right=1pt,
            top=1pt,
            bottom=1pt,
            boxrule=0pt}


\makeatletter

% ------------------------------- SECTIONING COMMANDS CUSTOMIZATION -------------------------------

% section: add optional argument to command for script page numbers
\let\old@sec\section%
\RenewDocumentCommand{\section}{somg}{%
    \IfBooleanTF{#1}{
        \IfNoValueTF{#2}{
            \IfNoValueTF{#4}{
                \old@sec*{#3}
            }{
                \old@sec*{#3 {\small(S. #4)}}
            }
        }{
            \IfNoValueTF{#4}{
                \old@sec*[#2]{#3}
            }{
                \old@sec*[#2]{#3 {\small(S. #4)}}
            }
        }%
    }{
        \IfNoValueTF{#2}{
            \IfNoValueTF{#4}{
                \old@sec{#3}
            }{
                \old@sec{#3 {\small(S. #4)}}
            }
        }{
            \IfNoValueTF{#4}{
                \old@sec[#2]{#3}
            }{
                \old@sec[#2]{#3 {\small(S. #4)}}
            }
        }%
    }
}


% subsection: add optional argument to command for script page numbers
\let\old@subsec\subsection%
\RenewDocumentCommand{\subsection}{somg}{%
    \IfBooleanTF{#1}{
        \IfNoValueTF{#2}{
            \IfNoValueTF{#4}{
                \old@subsec*{#3}
            }{
                \old@subsec*{#3 {\small(S. #4)}}
            }
        }{
            \IfNoValueTF{#4}{
                \old@subsec*[#2]{#3}
            }{
                \old@subsec*[#2]{#3 {\small(S. #4)}}
            }
        }%
    }{
        \IfNoValueTF{#2}{
            \IfNoValueTF{#4}{
                \old@subsec{#3}
            }{
                \old@subsec{#3 {\small(S. #4)}}
            }
        }{
            \IfNoValueTF{#4}{
                \old@subsec[#2]{#3}
            }{
                \old@subsec[#2]{#3 {\small(S. #4)}}
            }
        }%
    }
}


% subsubsection: add optional argument to command for script page numbers
\let\old@subsubsec\subsubsection%
\RenewDocumentCommand{\subsubsection}{somg}{%
    \IfBooleanTF{#1}{
        \IfNoValueTF{#2}{
            \IfNoValueTF{#4}{
                \old@subsubsec*{#3}
            }{
                \old@subsubsec*{#3 {\small(S. #4)}}
            }
        }{
            \IfNoValueTF{#4}{
                \old@subsubsec*[#2]{#3}
            }{
                \old@subsubsec*[#2]{#3 {\small(S. #4)}}
            }
        }%
    }{
        \IfNoValueTF{#2}{
            \IfNoValueTF{#4}{
                \old@subsubsec{#3}
            }{
                \old@subsubsec{#3 {\small(S. #4)}}
            }
        }{
            \IfNoValueTF{#4}{
                \old@subsubsec[#2]{#3}
            }{
                \old@subsubsec[#2]{#3 {\small(S. #4)}}
            }
        }%
    }
}


% custom text rightarrow to match tikz arrows
\renewcommand{\textrightarrow}{
    \tikz{
        \draw[-{Stealth[length=1.7mm]},
              double]
                (0,0) to (0.3,0);}}

% custom text leftrightarrow to match tikz arrows
\newcommand{\textlrarrow}{
    \tikz{
        \draw[{Stealth[length=1.7mm]}-{Stealth[length=1.7mm]},
              double]
                (0,0) to (0.4,0);}}


% renews the pmatrix environment to use \lgroup and \rgroup instead of \left( and \right)
\renewenvironment{pmatrix}{%
    \left\lgroup%
    \matrix@check\pmatrix\env@matrix%
}{
    \endmatrix\right\rgroup%
}

% renews the pmatrix* environment to use \lgroup and \rgroup instead of \left( and \right)
\MHInternalSyntaxOn%
\renewenvironment{pmatrix*}[1][c]
  {\left\lgroup\MT_matrix_begin:N #1}
  {\MT_matrix_end:\right\rgroup}
\MHInternalSyntaxOff%


% custom command for size matched colored brackets
\newcommand{\bbr}[2]{\colorlet{saved}{.}\color{#1}\left\lgroup\color{saved}#2\color{#1}\right\rgroup\color{saved}}


% custom command for differential operator d
\newcommand{\diff}{\ensuremath{\mathop{} \! \mathrm{d}}}

\newcommand{\rreal}{\ensuremath{\mathbb{R}}}
\newcommand{\nnatural}{\ensuremath{\mathbb{N}}}

% custom command for underset limes operator
\newcommand{\limes}[1]{\ensuremath{\underset{#1}{\lim}}}

% custom command for absolute value
\newcommand{\abs}[1]{\ensuremath{\left|#1\right|}}


% shortcuts for colored text
\newcommand{\cgn}[1]{{\color{green}#1}}
\newcommand{\crd}[1]{{\color{red}#1}}
\newcommand{\cbl}[1]{{\color{blue}#1}}
\newcommand{\cor}[1]{{\color{orange}#1}}
\newcommand{\cvt}[1]{{\color{violet}#1}}


\newcommand{\warn}[1]{
    \tikz[baseline=($(current bounding box.center)!0.65!(current bounding box.south)$)]{
        \node[isosceles triangle, 
              isosceles triangle stretches, 
              shape border rotate=90, 
              draw=red, 
              text=red,
              semithick,
              inner sep=0mm,
              minimum width=3mm, 
              minimum height=2.5mm, 
              rounded corners=0.5mm] {\bfseries \raisebox{0.1mm}{\footnotesize !}};}\crd{\textbf{ #1}}}

% \newcommand{\warnsymbol}{
%     \hspace{0.5mm}
%     \tikz[baseline=($(current bounding box.center)!0.65!(current bounding box.south)$)]{
%         \node[isosceles triangle, 
%               isosceles triangle stretches, 
%               shape border rotate=90, 
%               draw=red, 
%               text=red,
%               semithick,
%               inner sep=0mm,
%               minimum width=3mm, 
%               minimum height=2.5mm, 
%               rounded corners=0.5mm] {\bfseries \raisebox{0.1mm}{\footnotesize !}};}
% }


\newcommand{\txtqt}[1]{\textquotedbl #1\textquotedbl}

% bullet command for items in tables
\newcommand{\tabitem}{~~\llap{\textbullet}~~}


% customizes watermark and page color after a certain page limit
% colors all pages after the specified limit red
% source: https://stackoverflow.com/questions/2720534/force-a-maximum-number-of-pages-in-latex 
\newcounter{page@count}
\setcounter{page@count}{0}
\gdef\maxpages{\pagelimit}
\ifx\latex@outputpage\@undefined\relax% ChkTeX 21
    \global\let\latex@outputpage\@outputpage% ChkTeX 21
\fi%
\gdef\@outputpage{% ChkTeX 21
    \addtocounter{page@count}{1}%
    \ifnum\value{page@count}>\maxpages\relax%
        % change page background to red and add watermark
        \SetWatermarkText{\pagelimit\ Seiten Limit erreicht!}%
        \SetWatermarkScale{0.35}%
        \pagecolor{red}
        \latex@outputpage%
    \else%
        \SetWatermarkText{}%
        \latex@outputpage%
    \fi%
}


% remove title from table of contents, needed for layout
\renewcommand{\tableofcontents}{%
    \@starttoc{toc}
}


% scale super- and subscript -- not used currently, instead resized math font
% \catcode`_=\active% chktex 41 --> suppress ChkTeX warning
% \catcode`^=\active% chktex 41
% \newcommand_[1]{\ensuremath{\sb{\mathrm{\scaleobj{0.7}{#1}}}}}
% \newcommand^[1]{\ensuremath{\sp{\mathrm{\scaleobj{0.7}{#1}}}}}


\makeatother


% new environment for layout --> automatically adjusts to landscape or portrait
\NewDocumentEnvironment{layout}{}
                            {\ifthenelse{\paperwidth > \paperheight} % ChkTeX 1
                            % LANDSCAPE LAYOUT
                            {\begin{multicols*}{3} % ChkTeX 15
                                \begin{minipage}[t]{0.18\columnwidth}
                                    \vspace{-0.225\columnwidth}
                                    \qrcode[level=L, 
                                            version=0,
                                            height=0.9\columnwidth]{\repo}\\[1mm]
                                    \normalfont\footnotesize V \version{}
                                    \smallskip
                                \end{minipage}\hfill
                                \begin{minipage}[t]{0.81\columnwidth}
                                    \raggedright%
                                    \normalfont\huge\bfseries\title{}\\
                                    \normalfont\Large\semester\ -- \dozent{}\\
                                    \normalfont\large Autoren:\\
                                    \normalfont\large\author{}\\[2mm]
                                \end{minipage}
                                \normalfont\normalsize\myul{\url{\repo}}
                                \raggedcolumns}
                            % PORTRAIT LAYOUT
                            {\hfill\null\vspace{1cm}
                            \begin{center}
                                \normalfont\fontsize{35}{32}\selectfont\bfseries\title{}\\[7.5mm]
                                \normalfont\huge\semester\ \dozent{}\\
                                \normalfont\Large Autoren:\\
                                \normalfont\Large\author{}\\[2mm]
                                \normalfont\large Version:\\
                                \normalfont\large\version{}\\
                                \normalfont\normalsize\myul{\url{\repo}}\\[2mm]
                                \qrcode[level=L, 
                                        version=0,
                                        height=2cm]{\repo}
                            \end{center}
                            \vfill
                            \section*{\contentsname}
                            \begingroup
                            \setlength{\columnsep}{5mm}
                            \begin{multicols}{2}
                                \tableofcontents%
                            \end{multicols}
                            \endgroup
                            \vfill
                            \thispagestyle{empty}
                            \newpage
                            \begin{multicols*}{2}
                            \raggedcolumns}
                         }
                         {\end{multicols*}}


% new environment for centered tabulars
\NewDocumentEnvironment{ctabular}{m}
                        {\center\tabular{#1}}
                        {\endtabular\endcenter}


\NewDocumentEnvironment{rrcases}{}
                        {\left.\begin{aligned}}
                        {\end{aligned}\right\rbrace}
% ---------------------------------------- LISTINGS SETUP -----------------------------------------

% % hack to fix asterisk in lstlisting
% \makeatletter
% \lst@CCPutMacro%
%     \lst@ProcessOther{"2A}{% ChkTeX 18
%          {\raisebox{0.125pt}{*}}}
%     \@empty\z@\@empty% ChkTeX 21
% \makeatother


% % inline lst tikz node for later referencing
% \newcommand{\lstnode}[2][0.5ex]{
% 	\tikz[overlay, remember picture, inner sep=0pt, yshift=#1, minimum width=0mm]\node(#2){};
% }


% custom inline listings with box around them
% \newcommand{\mylstbox}[2][columns=fullflexible]{
%     \mybox{
%         \lstinline[#1]{#2}}}
% \newcommand{\mytclstbox}[2][columns=fullflexible]{
%     \mybox{
%         \lstinline[basicstyle=\sffamily\footnotesize\color{#1}, columns=fullflexible]{#2}}}


% % renew command for lstinputlisting with less vertical spacing
% \renewcommand{\lstinputlisting}[2][]{
%     \begingroup
%     \vspace{-0.6\abovedisplayskip}
%     \lst@setcatcodes%
%     \lst@inputlisting[#1]{#2}
%     \vspace{-0.6\abovedisplayskip}
% }


% % listings style (code)
% \lstdefinestyle{mystyle}{
%     backgroundcolor=\color{backcolour},   
%     commentstyle=\color{commentcolour},
%     keywordstyle=\bfseries\color{keywordcolour},
%     numberstyle=\tiny\color{numbercolour},
%     stringstyle=\color{stringcolour},
%     basicstyle=\sffamily,
%     breakatwhitespace=false,
%     breaklines=true,
%     captionpos=b,
%     keepspaces=true,
%     numbers=left,
%     numbersep=2pt,
%     showspaces=false,
%     showstringspaces=false,
%     showtabs=false,
%     tabsize=4,
% 	xleftmargin=1em,
% 	language=Java,
%     extendedchars=true,
%     inputencoding=cp1252,
% 	columns=[l]fullflexible	% see: https://tex.stackexchange.com/questions/99416/latex-source-code-listing-with-less-space-between-characters or manual
% }


% % custom lstinline command with custom colors
% \def\purplst{\begingroup\color{violet}}
% \def\greenlst{\begingroup\color{green}}
% \def\redlst{\begingroup\color{red}}
% \def\ywlst{\begingroup\color{orange}}
% \def\bluelst{\begingroup\color{blue}}
% \def\endlstcol{\endgroup}


% % basic lstinline style without colors
% \lstdefinestyle{basestyle}{
%     backgroundcolor=\color{backcolour},
%     keywordstyle=\bfseries,
%     numberstyle=\tiny\color{numbercolour},
%     basicstyle=\sffamily\footnotesize,
%     breakatwhitespace=false,     
%     breaklines=true,
%     captionpos=b,
%     keepspaces=true,
%     numbers=none,
%     numbersep=2pt,
%     showspaces=false,
%     showstringspaces=false,
%     showtabs=false,
%     tabsize=4,
%     xleftmargin=0em,
%     language=Java,
%     columns=flexible,	% see: https://tex.stackexchange.com/questions/99416/latex-source-code-listing-with-less-space-between-characters or manual
% }


% \lstset{
%     style=mystyle,
%     morekeywords={final, override, enum, var, List, Set, Map, String, Object},
%     moredelim=[il][\textcolor{orange}]{¦¦},
%     moredelim=[is][\textcolor{orange}]{&&}{&&},
%     moredelim=[is][\textcolor{violet}]{@}{@},
% }


% =========================================== DOCUMENT ============================================
\begin{document}
    \begin{layout}
        % \section{Dimensionen, Schnitte und Konturen}
        % \newpage
\section{Ableitungen, DGL und Gradienten (bi-variat)}
\[
    f:\mathbb{D}_f\subseteq \mathbb{R}^2\rightarrow \mathbb{W}_f\subseteq \mathbb{R}\quad\mathrm{skalar}
\]
\subsection{Partielle Ableitung}
Ableitung einer Partiellen Funktion. 

\example{Bi-Variate Funktion}
\[
    f(x, y):\, y \text{ fixieren} = \text{const.} = y_{0};\quad x \text{ \textbf{einzige} freie Variable}
\]

\subsubsection*{Notationen}

\begin{ctabular}{l l}
    1. Ordnung: & $\displaystyle f(x; y_{0})\Rightarrow \frac{\partial f}{\partial x} = f_{x}(x; y_{0})$ \\
    \multirow{2}*{2. Ordnung:}  &  $\displaystyle \frac{\partial}{\partial x}
                                    \left\lgroup\frac{\partial f}{\partial x}\right\rgroup = 
                                    \frac{\partial^{2}f}{\partial x^{2}} = f_{xx}$\\
                                &  $\displaystyle \frac{\partial}{\partial y}
                                        \left\lgroup\frac{\partial f}{\partial x}\right\rgroup = 
                                    \frac{\partial^{2}f}{\partial y \partial x} = f_{xy}$
\end{ctabular}


\subsubsection{Schwarz-Symmetrie}
Wenn \cor{$f_{xx}, f_{yy}, f_{xy}$ \& $f_{yx}$} \textbf{stetig} (sprungfrei) sind, dann gilt:
\[
    f_{xy} \overset{!}{=} f_{yx}
\]

         
\subsection{Gradient (Nabla-Operator)}\label{section:diff_dgl_gradient_bivar:gradient}
Spaltenvektor mit partiellen Ableitungen
\[
    \tikznode{grad}{\nabla} f = \begin{pmatrix}
        \frac{\partial f^{\mathstrut}}{\partial^{\mathstrut} x}\\
        \frac{\partial f^{\mathstrut}}{\partial^{\mathstrut} y}\\
        \vdots      % TODO: reformat vdots. They look like shit
    \end{pmatrix} \hat{=} \text{Vektorfeld}
\]

% TODO: some TikZ. Might not be needed and could be removed if unnecessary
\begin{tikzpicture}[overlay, remember picture, >={Latex}]
    \node[inner sep=0pt, ellipse, fit=(grad), draw, densely dotted]{};
    \node[below left= 2mm and 5mm of grad] (gtext) {\txtqt{Gradient} / Nabla};
    \draw[->, rounded corners] (gtext.east) -| ($(grad.south)+(0,-1mm)$);
\end{tikzpicture}


\subsection{Totale Ableitung}
Für Fehlerrechnung benützt, da man hierbei die Abstände von $(x; y; z)$ 
zu einem festen Punkt $(x_{0}; y_{0}; z_{0})$ erhält. (relative Koordinaten)

\[
    D(f;\underbrace{(x_{0}, y_{0}, \ldots)}_{\text{Arbeitspunkt}}):\, 
    \rreal^{\tikznode{somea}{\scriptstyle 2}} \rightarrow \rreal^{\tikznode{someb}{\scriptstyle 1}};\, 
    \text{\txtqt{gute Approximation}}
\]
\[
    f(x = x_{0} + \Delta x; y = y_{0} + \Delta y; \ldots) = (D_{11}; D_{12})\cdot \begin{pmatrix}
        \Delta x\\
        \Delta y
    \end{pmatrix} + f(x_{0}; y_{0}) + \cgn{R_{1}}
\]
Wobei \cgn{$R_{1}$} dem \txtqt{Rest} entspricht. (Ähnlich wie bei Taylorreihe)

\begin{minipage}[c]{0.7\columnwidth}
    \[
        \frac{\cgn{R_{1}}}{\cor{d = \sqrt{\Delta x^{2} + \Delta y^{2}}}} \rightarrow 0 \text{ (\txtqt{gut}, \txtqt{schneller gegen 0 als $\cor{d}$})}
    \]
\end{minipage}\hfill
\begin{minipage}[c]{0.29\columnwidth}
    \begin{tikzpicture}[baseline=(current bounding box.center), 
                        >={Latex[width=1mm, 
                                 length=1mm]}, 
                        scale=0.5, 
                        font=\tiny]

        % Koordinatensystem
        \draw[->] (0,0) -- (4,0) node[below]{$x$};
        \draw[->] (0,0) -- (0,2.5) node[left]{$y$};

        % Punkte
        \node[circle, fill=black, inner sep=0pt, minimum size=2pt] (A) at (1.7,0.7) {};
        \node[circle, fill=black, inner sep=0pt, minimum size=2pt] (P) at (0.7,1.7) {};
        \node[inner sep=1pt, right=1mm of A] {$A=(x_{0}; y_{0})$};
        \node[inner sep=1pt, above right=0.1mm and -1ex of P] {$P=(x;y)$};
    
        % Differenz
        \draw (P) -- (A-|P) node[midway, left]{$\Delta y$} -- (A) node[midway, below]{$\Delta x$};
        \draw[orange] (P) -- (A) node[midway, above right, inner sep=1pt]{\cor{$d$}};
    \end{tikzpicture}
\end{minipage}

\[
    \begin{split}
        D(f;(x_{0}; y_{0})) &= \left\lgroup D_{11} = \frac{\partial f}{\partial x}(x_{0}; y_{0}); 
                                            D_{12} = \frac{\partial f}{\partial y}(x_{0}; y_{0})\right\rgroup\\
        &= (\nabla f)^{\tr} \text{ \cbl{wenn $\frac{\partial f}{\partial x}; \frac{\partial f}{\partial y}$ stetig bei $A$}}
    \end{split}
\]

\begin{tikzpicture}[overlay, remember picture]
    \draw[tips, -{Latex}] (someb) to[out=135, in=45, edge node={node[above=0mm]{\tiny $1\times 2$ Matrix}}] (somea);
\end{tikzpicture}


\subsection{Linearapproximation (Tangentialapproximation)}
\[
    f(x;y) \approx f(x_{0}; y_{0}) + D(f;(x_{0}; y_{0}))\cdot \begin{pmatrix}
        \Delta x\\
        \Delta y
    \end{pmatrix}
    \quad\text{ linear in $\Delta x$ und $\Delta y$}
\]


\subsubsection{Tangentialebene}
\[
    \crd{g(x; y) = f(x_{0}; y_{0}) + D(f;(x_{0}; y_{0}))\cdot \begin{pmatrix}
        x - x_{0}\\
        y - y_{0}
    \end{pmatrix}}
\]
\[
    g(x;y)=f(x_0;y_0)+f_x(x_0;y_0)\cdot(x-x_0)+f_y(x_0;y_0)\cdot(y-y_0)
\]

\subsubsection{Tangentialer Anstieg (Totale Differential)}
\[
    \cvt{\diff f \overset{!}{=} 
        \frac{\partial f}{\partial \tikznode{ptx}{x}}\diff x + 
        \frac{\partial f}{\partial \tikznode{pty}{y}}\diff y} 
        \quad \text{bezüglich } A=\underbrace{(x_{0}; y_{0})}_{\tikznode{axy}{}}
\]
\begin{tikzpicture}[overlay, 
                    remember picture, 
                    >={Latex[width=1mm, 
                             length=1mm]}]

    \draw[->] ($(axy)+(0, 1.8mm)$) to[bend left=10] (ptx.south east);
    \draw[->] ($(axy)+(0, 1.8mm)$) to[bend left=10] (pty.south east);
\end{tikzpicture}


\subsubsection{Differential-Trick (\texorpdfstring{$\diff f$}{df} Trick)}
Auf Kontouren sei $\diff f = 0$ (Kontourlinien). Daher lässt sich folgende Gleichung aufstellen:
\[
\begin{aligned}
    f &= c = \mathrm{const.} \quad | \diff(\ldots)\\
    \diff f &= \diff c \overset{!}{=} 0
\end{aligned}
\]
Bzw. für Kontourlinien: $f_{x}\diff x + f_{y}\diff y = 0$


\subsubsection{Implizite (Steigungs-)Funktion}
\begin{minipage}[c]{0.6\columnwidth}
    \[
        \cbl{y'(x)} = \frac{\diff y}{\diff x} = -\frac{f_{x}}{f_{y}\crd{\neq 0}} \lor 
        \cbl{x'(y)} = \frac{\diff x}{\diff y} = -\frac{f_{y}}{f_{x}\crd{\neq 0}}
    \]
\end{minipage}\hfill
\begin{minipage}[c]{0.39\columnwidth}
    \begin{tikzpicture}[>={Latex[width=1mm,
                                 length=1mm]}, 
                        scale=0.75, 
                        font=\small]

        % Koordinatensystem
        \draw[->] (0,0) -- (3,0) node[below]{$x$};
        \draw[->] (0,0) -- (0,2) node[left]{$y$};
        
        % Funktionen
        \draw[color=green] plot[domain=0.55:1.5] (\x, {-((\x-1.5)*(\x-1.5))+1}) node[right]{$y$}; % y = -((x-1.5)^2)+1
        \draw[color=blue] node[above right]{$y'$} plot[domain=0.25:1.5] (\x, {\x-0.25}); % y = x-0.25

        % Arbeitspunkt
        \node[circle, fill=black, inner sep=0pt, minimum size=1.5pt] (P) at (1,0.75) {};
        \node[inner sep=0pt, above left=0.25mm and 0mm of P] {$P_{0}$};
        \draw[gray, dashed] (P) -- (P-|0,0) node[left]{$y_{0}$};
        \draw[gray, dashed] (P) -- (P|-0,0) node[below]{$x_{0}$};

        % Richtungselemente
        \draw[orange, semithick, ->] (P) -- (2,1.75) node[above, midway, rotate=45]{$\vec{r}$};
        \draw[->] (P) -- (P-|2,0) node[midway, below]{$\diff x$};
        \draw[->] (P-|2,0) -- (2,1.75) node[midway, right]{$\diff y$};
    \end{tikzpicture}
\end{minipage}




% \subsection{Differential}


\subsection{DGL}
\[
    y' = \tikznode{rhsfn}{\bbr{violet}{-\frac{f_{x}}{f_{y}}}};\, y(x_{0}) = y_{0}
\]
\begin{tikzpicture}[overlay, 
                    remember picture] 
    \node[below=1mm of rhsfn, inner sep=0pt, font=\tiny, text=violet] {right-hand-side (r.h.s.) Funktion};
\end{tikzpicture}


\subsection{Richtungselement (Tangentiallinie an Kontouren)}
\[
    \vec{r} = \left\lgroup\diff x = h; \diff y = y'\diff x = -\frac{f_{x}}{f_{y}}\diff x\right\rgroup^{\tr}
\]


\subsection{Gradientenfeld \texorpdfstring{$\perp$}{\_|\_} Kontouren}
\[
    \nabla f \tikznode{scprd}{\dotp} \begin{pmatrix}
        \diff x\\
        \diff y = y'\diff x
    \end{pmatrix} \overset{!}{=} 0
\]
\begin{tikzpicture}[overlay,
                    remember picture,
                    >={Latex[width=1mm, 
                             length=1mm]}]
    \node[above left=2mm and 4mm of scprd, inner sep=0pt, anchor=south east] (spnode) {Skalarprodukt};
    \draw[->] (spnode.east) to[out=0, in=90] ($(scprd.north)+(0,0.2mm)$);
\end{tikzpicture}


\subsection{TODO: Taylorreihe}

\begin{tikzpicture}[>={Latex[width=1mm,
                       length=1mm]}]
    \draw[->] (-1,0) -- (3,0) node[below]{$x$};
    \draw[->] (0,-0.5) -- (0,2) node[left]{$y$};

    % Schnitt
    \draw[green] (-1,-0.5) -- (1, 0.75);
\end{tikzpicture}


\begin{ctabular}{r l}
    $s(t) :$    &$P_0 + t \cdot \hat{v} \mid t\in \rreal$\\
    $s(t) :$    &$f(x_0 + t \cdot \hat{v}_1\, ;\, y_0 + t \cdot \hat{v}_2)$\\
    &\\
    $\frac{\diff s(t)}{\diff t}=\dot{s}(t) :$&$t\mapsto\overbrace{\begin{pmatrix}x_0+t\cdot v_1\\y_0+t\cdot v_2\end{pmatrix}}^{\begin{pmatrix}x\\y\end{pmatrix}}\mapsto f(x,y)$
\end{ctabular}


\subsection{Richtungs-Ableitung}
\[
    \frac{\partial f}{\partial\hat{v}}\overset{!}=D(f;(x_{0};y_{0}))\cdot\hat{v}\overset{\text{Def.}}\Leftrightarrow \grad(f)^{\tr}\cdot\hat{v}=f_{x}\cdot v_{1}+f_{y}\cdot v_{2}
\]

\example{Richtungs-Ableitung}

\[
    \vec{x} : \vec{v} = 
\begin{pmatrix}
    1\\
    0
\end{pmatrix}
= \hat{e}_1
\quad\Rightarrow\quad
    \frac{\partial f}{\partial\hat{e}_{1}}=f_{x}\cdot1+f_{y}\cdot0=\uuline{f_{x}\strut}
\]


\subsubsection{Spezialfälle}


\begin{minipage}[c]{.72\columnwidth}
    \begin{outline}
        \1 $\alpha = \frac{\pi}{2}$ \textrightarrow\ rechter Winkel % TODO: fix \pi --> should be upright
        \1 $\frac{\partial f}{\partial \hat{v}}$ extremal
            \2 $\alpha = 0$ (max):   $\nabla f \cdot\hat{v} > 0$ \textrightarrow\ \cbl{$\grad(f)$} liegt auf $\hat{v}$
            \2 $\alpha = \pi$ (min): $\nabla f \cdot\hat{v} < 0$ \textrightarrow\ \cbl{$\grad(f)$} liegt invers auf $\hat{v}$
    \end{outline}
\end{minipage}\hfill
\begin{tikzpicture}[%
    baseline=(current bounding box.center),
    scale=0.75,
    >={Latex[%
        width=1mm, 
        length=1mm]}]
    \coordinate (A) at (0,0);
    \coordinate (B1) at (2,0);
    \coordinate (B2) at (2.4,0);
    \coordinate (C) at (2,0.8);

    \draw[->] (A) -- (B2) node[right]{$\hat{v}$};
    \draw[blue, ->] (A) -- (C) node[above=-1mm]{$\grad(f)$};
    \draw[red, ->] (A) -- (B1) node[midway, below]{$\nabla f \dotp \hat{v}$};

    \draw[gray, densely dashed] (C) -- (B1);

    \draw pic[draw=green, angle radius=10mm, pic text=$\alpha$, pic text options=green] {angle=B1--A--C};
    \draw pic[draw=gray, angle radius=2mm, pic text=. , pic text options=gray] {angle=C--B1--A};

    \fill (A) circle (1pt) node[above]{$P_0$};
\end{tikzpicture}

\para{Trigo} $\nabla f\cdot\hat{v} \land \frac{\partial f}{\partial\hat{v}}$ \textrightarrow\ $\cos(\alpha)\cdot\abs{\cbl{\nabla f}}$


        % \newpage
\section{Extrema von Funktionen finden}
\myul{\textbf{Stationäritätsbedingung:}} $\nabla f \stackrel{!}{=} \vec{0}$

\subsection{Extrema von Funktionen zweier Variablen finden}
\begin{enumerate}[itemsep=1ex]
    \item \textbf{Gradient von $f$ Null-setzten und kritische Stellen finden:}

    $\nabla f=
    \begin{pmatrix}
        f_x\\
        f_y
    \end{pmatrix} \stackrel{!}{=}
    \begin{pmatrix}
        0\\
        0
    \end{pmatrix}
    \, \, \, \, \, \,
    \Rightarrow 
    \begin{matrix}
        f_{x}=0\\
        f_{y}=0
    \end{matrix}
    \, \, \, \, \, \,
    \Rightarrow
    x_0 \text{ und } y_0 \text{ bestimmen}$

    \item \textbf{Zweite Partielle Ableitungen bestimmen:}
        
    \begin{tabular}{lll}
        $f_{xx} = \dots$ & $f_{xy} = f_{yx} = \dots$ & $f_{yy} = \dots$ \\
    \end{tabular}

    \item \textbf{Determinante $\Delta$ der Hesse-Matrix H bestimmen:}
    
    $\Delta = f_{xx}(x_0;y_0) \cdot f_{yy}(x_0;y_0) - \left(f_{xy}(x_0;y_0)\right)^2 $

    \item \textbf{Auswertung:}
    
    \begin{tabular}{lllcl}
        \hline
        $\Delta > 0$ &AND& $f_{xx}(x_0;y_0) < 0$ &$\Longrightarrow$& $\text{lokales Maximum}$\\
        \hline
        $\Delta > 0$ &AND& $f_{yy}(x_0;y_0) < 0$ &$\Longrightarrow$& $\text{lokales Maximum}$\\
        \hline
        $\Delta > 0$ &AND& $f_{xx}(x_0;y_0) > 0$ &$\Longrightarrow$& $\text{lokales Minimum}$\\
        \hline
        $\Delta > 0$ &AND& $f_{yy}(x_0;y_0) > 0$ &$\Longrightarrow$& $\text{lokales Minimum}$\\
        \hline
        $\Delta < 0$ &&&$\Longrightarrow$& $\text{Sattelpunkt}$\\
        \hline
        $\Delta = 0$ &&&?& $\text{Multi-variate-Taylor-logik ...}$\\
        \hline
    \end{tabular}

\end{enumerate}

\subsection{Extrema von Funktionen mehrerer Variablen finden}

\begin{enumerate}[itemsep=1ex]
    \item \textbf{Gradient von $f$ Null-setzten und kritische Stellen finden:}
    
    $\nabla f=
    \begin{pmatrix}
        f_x\\
        f_y\\
        \vdots \\
        f_t
    \end{pmatrix} \stackrel{!}{=}
    \begin{pmatrix}
        0\\
        0\\
        \vdots \\
        0
    \end{pmatrix}
    \quad \Rightarrow x_0$, $y_0$, $\ldots$, $t_0$ bestimmen
    
    \item \textbf{Zweite Partielle Ableitungen für Hesse-Matrix H bestimmen:}
    
    \begin{minipage}[t]{0.48\columnwidth}
        $\mathbf{H}=\begin{pmatrix}
            f_{xx}&f_{xy}&\ldots &f_{xt}\\
            f_{yx}&f_{yy}&\ldots&f_{yt}\\
            \vdots &\vdots &\ddots &\vdots \\
            f_{tx}&f_{ty}&\ldots&f_{tt}
        \end{pmatrix}$
    \end{minipage}\hfill
    \begin{minipage}[c]{0.48\columnwidth}
        \begin{itemize}
            \item Symmetrien beachten!
            \item Nicht doppelt rechnen!
            \item[] \textrightarrow\ $f_{xt} = f_{tx}$
        \end{itemize}
    \end{minipage}

    \item \textbf{Hesse-Matrix H mit gefundenen Stellen füllen:}

    $\mathbf{H}(x_0,y_0,\ldots t_0)=
    \begin{pmatrix}
        f_{xx}(x_0,y_0,\ldots t_0)&f_{xy}(x_0,y_0, \ldots t_0)&\cdots &f_{xt}(x_0,y_0,\ldots t_0)\\
        f_{yx}(x_0,y_0,\ldots t_0)&f_{yy}(x_0,y_0, \ldots t_0)&\cdots &f_{yt}(x_0,y_0,\ldots t_0)\\
        \vdots &\vdots &\ddots &\vdots \\
        f_{tx}(x_0,y_0,\ldots t_0)&f_{ty}(x_0,y_0, \ldots t_0)&\cdots &f_{tt}(x_0,y_0,\ldots t_0)\end{pmatrix}$

    \item \textbf{Eigenwerte $\lambda_i$ der Hesse-Matrix bestimmen:}

    $\text{det}\left(\mathbf{H}(x_0,y_0,\ldots t_0) - \lambda \cdot \mathbf{E}\right)  = 0$

    Nullstellen $\lambda_i$ finden $\rightarrow  $ Eigenwerte

    \medskip
    \myul{Zur Erinnerung:}\\
    $\mathbf{E} = \begin{pmatrix}
        1&0&\ldots &0\\
        0&1&\ldots&0\\
        \vdots &\vdots &\ddots &\vdots \\
        0&0&\ldots&1\\
    \end{pmatrix}
    , \quad
    \lambda \cdot \mathbf{E} = \begin{pmatrix}
        \lambda&0&\ldots &0\\
        0&\lambda&\ldots&0\\
        \vdots &\vdots &\ddots &\vdots \\
        0&0&\ldots&\lambda\\
    \end{pmatrix}$

    \medskip
    $\mathbf{H}(x_0,y_0,\ldots t_0) - \lambda \cdot \mathbf{E} = \ldots \\
    \ldots  = 
    \begin{pmatrix}
        f_{xx}(x_0,y_0,\ldots t_0)- \lambda&f_{xy}(x_0,y_0, \ldots t_0)&\cdots &f_{xt}(x_0,y_0,\ldots t_0)\\
        f_{yx}(x_0,y_0,\ldots t_0)&f_{yy}(x_0,y_0, \ldots t_0)- \lambda&\cdots &f_{yt}(x_0,y_0,\ldots t_0)\\
        \vdots &\vdots &\ddots &\vdots \\
        f_{tx}(x_0,y_0,\ldots t_0)&f_{ty}(x_0,y_0, \ldots t_0)&\cdots &f_{tt}(x_0,y_0,\ldots t_0)- \lambda\end{pmatrix}$

    \item \textbf{Auswertung:}
    
    \begin{tabular}{lll}
        \hline
        $\lambda_i < 0 \,\,\,\forall i$ &$\Longrightarrow$& $\text{lokales Maximum}$\\
        \hline
        $\lambda_i > 0 \,\,\,\forall i$ &$\Longrightarrow$& $\text{lokales Minimum}$\\
        \hline
        $\lambda_i > 0\,$ und $\,\lambda_i < 0$ &$\Longrightarrow$& $\text{Sattelpunkt}$\\
        \hline
    \end{tabular}

    \medskip
    Erklärung:
    \begin{itemize}
        \item $\lambda_i < 0 \,\,\,\forall i$ $\Leftrightarrow $ Alle $\lambda_i$ sind negativ
        \item $\lambda_i > 0 \,\,\,\forall i$ $\Leftrightarrow $ Alle $\lambda_i$ sind positiv
    \end{itemize}
\end{enumerate}


\subsection{Lokales oder Globales Extremum}
Für eine beliebige die Funktion $f(x, y, \ldots  , t)$ gilt:

$\boxed{\begin{array}{llll}
    f(x,y,\ldots ,t)\leq M_{\max}&\forall(x,y,\ldots ,t)\in\mathbb{D}_f&\Rightarrow&\text{globales Maxinum}\\
    f(x,y,\ldots ,t)>M_{\max}&\exists(x,y,\ldots ,t)\in\mathbb{D}_f&\Rightarrow&\text{kein globales Maximum}\\
    \hline f(x,y,\ldots ,t)\geq M_{\min}&\forall(x,y,\ldots ,t)\in\mathbb{D}_f&\Rightarrow&\text{globales Minimum}\\
    f(x,y,\ldots ,t)<M_{\min}&\exists(x,y,\ldots ,t)\in\mathbb{D}_f&\Rightarrow&\text{kein globales Minimum}
\end{array}}$

\medskip
\begin{tabular}{ll}
    $M_{\max}$: &\text{grösstes lokales Maximum}\\
    $M_{\min}$: &\text{kleinstes lokales Minimum}
\end{tabular}




\subsection{Extrema von Funktionen zweier Variablen mit NB finden}

\begin{enumerate}[itemsep=1ex]
    \item \textbf{Nebenbedingung (NB) in Standartform bringen:}\\
    \begin{minipage}[t]{0.4\linewidth}
        Standartform: $n(x, y) \stackrel{!}{=} 0$
    \end{minipage}\hfill
    \begin{minipage}[t]{0.58\linewidth}
            \textcolor{gray}{Nebenbedingung: $x + y = 1$}\\
            \textcolor{gray}{Standartform der Nebenbedingung: $x + y - 1 = 0$}
    \end{minipage}


    \item \textbf{Lagrancge-Funktion $\mathcal{L}$ aufstellen:}\\
    $\mathcal{L}(x, y, \lambda) =
    f(x, y) + \lambda \cdot n(x, y) \,\,\,\,\,\,$ \textcolor{gray}{Am besten gleich ausmultiplizieren}
    

    \item \textbf{Gradient der Lagrancge-Funktion $\mathcal{L}$ Null-setzten und kritische Stellen finden:}\\
    $\nabla \mathcal{L}=
    \begin{pmatrix}
        \mathcal{L}_x\\
        \mathcal{L}_y\\
        \mathcal{L}_\lambda
    \end{pmatrix} \stackrel{!}{=}
    \begin{pmatrix}
        0\\
        0\\
        0
    \end{pmatrix}
    \, \, \, \, \, \,
    \Rightarrow 
    x_0 \text{ und } y_0 \text{ bestimmen}$
    \hfill

    \item \textbf{Zweite Partielle Ableitungen bestimmen:}\\
    \begin{minipage}[t]{0.4\linewidth}
        $\begin{aligned}
            \mathcal{L}_{\lambda \lambda} &\stackrel{!}{=} 0\\
            \mathcal{L}_{xx} &= \dots\\
            \mathcal{L}_{yy} &= \dots\\
        \end{aligned}$
    \end{minipage}\hfill
    \begin{minipage}[c]{0.58\linewidth}
        $\begin{aligned}
            \mathcal{L}_{\lambda x} &= \mathcal{L}_{x\lambda} = n_x = \dots\\
            \mathcal{L}_{\lambda y} &= \mathcal{L}_{y\lambda} = n_y =\dots\\
            \mathcal{L}_{xy} &= \mathcal{L}_{yx} = \dots\\
        \end{aligned}$
    \end{minipage}


    \item \textbf{Geränderte Hesse Matrix $\overline{\mathbf{H}}$ aufstellen und kritische Stellen einsetzen:}
    
    \begin{tabular}{lll}
        $\overline{\mathbf{H}}(x_0,y_0)$ &$=$&
        $\left(
            \begin{matrix}
                {{\mathcal{L}_{\lambda\lambda}(x_0, y_0)}}&{{\mathcal{L}_{\lambda x}(x_0, y_0)}}&{{\mathcal{L}_{\lambda y}(x_0, y_0)}}\\
                {{\mathcal{L}_{x\lambda}(x_0, y_0)}}&{{\mathcal{L}_{xx}(x_0, y_0)}}&{{\mathcal{L}_{xy}(x_0, y_0)}}\\
                {{\mathcal{L}_{y\lambda}(x_0, y_0)}}&{{\mathcal{L}_{yx}(x_0, y_0)}}&{{\mathcal{L}_{yy}(x_0, y_0)}}\\
            \end{matrix}
        \right)$\\
        &$=$&
        $\left(
            \begin{matrix}
                {{0}}&{{n_{x}(x_0, y_0)}}&{{n_{y}(x_0, y_0)}}\\
                {{n_{x}(x_0, y_0)}}&{{\mathcal{L}_{xx}(x_0, y_0)}}&{{\mathcal{L}_{xy}(x_0, y_0)}}\\
                {{n_{y}(x_0, y_0)}}&{{\mathcal{L}_{yx}(x_0, y_0)}}&{{\mathcal{L}_{yy}(x_0, y_0)}}\\
            \end{matrix}
        \right)$
        
    \end{tabular}


    \item \textbf{Determinante der geränderten Hesse Matrix bestimmen:}\\
    $\text{det}\left(\overline{\mathbf{H}}\right) = ... $

    \item \textbf{Auswertung}\\
    \begin{tabular}{lll}
        \hline
        $\text{det}\left(\overline{\mathbf{H}}\right) > 0$ &$\Longrightarrow$& $\text{lokales Maximum}$\\
        \hline
        $\text{det}\left(\overline{\mathbf{H}}\right) < 0$ &$\Longrightarrow$& $\text{lokales Minimum}$\\
        \hline
        $\text{det}\left(\overline{\mathbf{H}}\right) = 0$ &$\Longrightarrow$& $\text{keine Aussage möglich}$\\
        \hline
    \end{tabular}

\end{enumerate}

\subsection{Extrema von Funktionen mehrerer Variablen mit NB finden}

\begin{enumerate}[itemsep=1ex]
    \item \textbf{Nebenbedingung (NB) in Standartform bringen:}\\
    \begin{minipage}[t]{0.4\columnwidth}
        Standartform: $n(x, y, ... , t) \stackrel{!}{=} 0$
    \end{minipage}\hfill
    \begin{minipage}[t]{0.6\columnwidth}
    \end{minipage}


    \item \textbf{Lagrancge-Funktion $\mathcal{L}$ aufstellen:}\\
    $\mathcal{L}(x, y, ..., t, \lambda) =
    f(x, y, ..., t) + \lambda \cdot n(x, y, ..., t) \,\,\,\,\,\,$ \textcolor{gray}{Am besten gleich ausmultiplizieren}
    

    \item \textbf{Gradient der Lagrancge-Funktion $\mathcal{L}$ Null-setzten und kritische Stellen finden:}
    $\nabla \mathcal{L}=
    \begin{pmatrix}
        \mathcal{L}_x\\
        \mathcal{L}_y\\
        \vdots \\
        \mathcal{L}_t\\
        \mathcal{L}_\lambda
    \end{pmatrix} \stackrel{!}{=}
    \begin{pmatrix}
        0\\
        0\\
        \vdots \\
        0\\
        0
    \end{pmatrix}
    \, \, \, \, \, \,
    \Rightarrow 
    x_0 \text{, } y_0 \text{, }... \text{, } t_0 \text{ bestimmen}$
    \hfill

    \item \textbf{Zweite Partielle Ableitungen bestimmen:}
    

    \begin{tabular}{lllll}
        $\begin{matrix}
        \mathcal{L}_{\lambda \lambda} \stackrel{!}{=} 0\\
        \mathcal{L}_{xx} = \dots\\
        \mathcal{L}_{yy} = \dots\\
        \vdots \\
        \mathcal{L}_{tt} = \dots\\
        \end{matrix}$
        &$\,\,\,\,\,\,\,\,$&
        $\begin{matrix}
        \mathcal{L}_{\lambda x} = \mathcal{L}_{x\lambda} = n_x = \dots\\
        \mathcal{L}_{\lambda y} = \mathcal{L}_{y\lambda} = n_y =\dots\\
        \vdots \\
        \mathcal{L}_{\lambda t} = \mathcal{L}_{t\lambda} = n_t =\dots\\
        \end{matrix}$
        &$\,\,\,\,\,\,\,\,$&
        $\begin{matrix}
        \mathcal{L}_{xy} = \mathcal{L}_{yx}\\
        \mathcal{L}_{xt} = \mathcal{L}_{tx}\\
        \mathcal{L}_{yt} = \mathcal{L}_{ty}\\
        \vdots \\
        \end{matrix}$
    \end{tabular}

    \item \textbf{Geränderte Hesse Matrix $\overline{\mathbf{H}}$ aufstellen und kritische Stellen einsetzen:}\\
    \begin{tabular}{lll}
        $\overline{\mathbf{H}}(\crd{x_0,y_0,\ldots t_0})$ &$=$&
        $\left(
            \begin{matrix}
                {{\mathcal{L}_{\lambda\lambda}(\crd{...})}}&{{\mathcal{L}_{\lambda x}(\crd{...})}}&{{\mathcal{L}_{\lambda y}(\crd{...})}}&{\cdots }&{{\mathcal{L}_{\lambda t}(\crd{...})}}\\
                {{\mathcal{L}_{x\lambda}(\crd{...})}}&{{\mathcal{L}_{xx}(\crd{...})}}&{{\mathcal{L}_{xy}(\crd{...})}}&{\cdots }&{{\mathcal{L}_{xt}(\crd{...})}}\\
                {{\mathcal{L}_{y\lambda}(\crd{...})}}&{{\mathcal{L}_{yx}(\crd{...})}}&{{\mathcal{L}_{yy}(\crd{...})}}&{\cdots }&{{\mathcal{L}_{yt}(\crd{...})}}\\
                {\vdots }&{\vdots }&{\vdots }&{\ddots }&{\vdots }\\
                {{\mathcal{L}_{t\lambda}(\crd{...})}}&{{\mathcal{L}_{tx}(\crd{...})}}&{{\mathcal{L}_{ty}(\crd{...})}}&{\cdots }&{{\mathcal{L}_{tt}(\crd{...})}}\\
            \end{matrix}
        \right)$\\
        &$=$&
        $\left(
            \begin{matrix}
                {{0}}&{{n_{x}(\crd{...})}}&{{n_{y}(\crd{...})}}&{\cdots }&{{n_{t}(\crd{...})}}\\
                {{n_{x}(\crd{...})}}&{{\mathcal{L}_{xx}(\crd{...})}}&{{\mathcal{L}_{\lambda y}(\crd{...})}}&{\cdots }&{{\mathcal{L}_{xt}(\crd{...})}}\\
                {{n_{y}(\crd{...})}}&{{\mathcal{L}_{yx}(\crd{...})}}&{{\mathcal{L}_{yy}(\crd{...})}}&{\cdots }&{{\mathcal{L}_{yt}(\crd{...})}}\\
                {\vdots }&{\vdots }&{\vdots }&{\ddots }&{\vdots }\\
                {{n_{t}(\crd{...})}}&{{\mathcal{L}_{tx}(\crd{...})}}&{{\mathcal{L}_{ty}(\crd{...})}}&{\cdots }&{{\mathcal{L}_{tt}(\crd{...})}}\\
            \end{matrix}
        \right)$
    \end{tabular}

    \item \textbf{Determinante der geränderten Hesse Matrix bestimmen:}\\
    $\text{det}\left(\overline{\mathbf{H}}\right) = ... $

    \item \textbf{Auswertung}\\
    \begin{tabular}{lll}
        \hline
        $\text{det}\left(\overline{\mathbf{H}}\right) > 0$ &$\Longrightarrow$& $\text{lokales Maximum}$\\
        \hline
        $\text{det}\left(\overline{\mathbf{H}}\right) < 0$ &$\Longrightarrow$& $\text{lokales Minimum}$\\
        \hline
        $\text{det}\left(\overline{\mathbf{H}}\right) = 0$ &$\Longrightarrow$& $\text{keine Aussage möglich}$\\
        \hline
    \end{tabular}

\end{enumerate}
        % \newpage
\section{Extrema von Funktionen mehrerer Variabeln finden}

\begin{enumerate}[itemsep=1ex]
    \item \textbf{Gradient von $f$ Null-setzten und kritische Stellen finden:}
    
    $\nabla f=
    \begin{pmatrix}
        f_x\\
        f_y\\
        \vdots \\
        f_t
    \end{pmatrix} \stackrel{!}{=}
    \begin{pmatrix}
        0\\
        0\\
        \vdots \\
        0
    \end{pmatrix}
    \quad \Rightarrow x_0$, $y_0$, $\ldots$, $t_0$ bestimmen
    
    \item \textbf{Zweite Partielle Ableitungen für Hesse-Matrix H bestimmen:}
    
    \begin{minipage}[t]{0.48\columnwidth}
        $\mathbf{H}=\begin{pmatrix}
            f_{xx}&f_{xy}&\ldots &f_{xt}\\
            f_{yx}&f_{yy}&\ldots&f_{yt}\\
            \vdots &\vdots &\ddots &\vdots \\
            f_{tx}&f_{ty}&\ldots&f_{tt}
        \end{pmatrix}$
    \end{minipage}\hfill
    \begin{minipage}[c]{0.48\columnwidth}
        \begin{itemize}
            \item Symmetrien beachten!
            \item Nicht doppelt rechnen!
            \item[] \textrightarrow\ $f_{xt} = f_{tx}$
        \end{itemize}
    \end{minipage}

    \item \textbf{Hesse-Matrix H mit gefundenen Stellen füllen:}

    $\mathbf{H}(x_0,y_0,\ldots t_0)=
    \begin{pmatrix}
        f_{xx}(x_0,y_0,\ldots t_0)&f_{xy}(x_0,y_0, \ldots t_0)&\cdots &f_{xt}(x_0,y_0,\ldots t_0)\\
        f_{yx}(x_0,y_0,\ldots t_0)&f_{yy}(x_0,y_0, \ldots t_0)&\cdots &f_{yt}(x_0,y_0,\ldots t_0)\\
        \vdots &\vdots &\ddots &\vdots \\
        f_{tx}(x_0,y_0,\ldots t_0)&f_{ty}(x_0,y_0, \ldots t_0)&\cdots &f_{tt}(x_0,y_0,\ldots t_0)\end{pmatrix}$

    \item \textbf{Eigenwerte $\lambda_i$ der Hesse-Matrix bestimmen:}

    $\text{det}\left(\mathbf{H}(x_0,y_0,\ldots t_0) - \lambda \cdot \mathbf{E}\right)  = 0$

    Nullstellen $\lambda_i$ finden $\rightarrow  $ Eigenwerte

    \medskip
    \myul{Zur Erinnerung:}\\
    $\mathbf{E} = \begin{pmatrix}
        1&0&\ldots &0\\
        0&1&\ldots&0\\
        \vdots &\vdots &\ddots &\vdots \\
        0&0&\ldots&1\\
    \end{pmatrix}
    , \quad
    \lambda \cdot \mathbf{E} = \begin{pmatrix}
        \lambda&0&\ldots &0\\
        0&\lambda&\ldots&0\\
        \vdots &\vdots &\ddots &\vdots \\
        0&0&\ldots&\lambda\\
    \end{pmatrix}$

    \medskip
    $\mathbf{H}(x_0,y_0,\ldots t_0) - \lambda \cdot \mathbf{E} = \ldots \\
    \ldots  = 
    \begin{pmatrix}
        f_{xx}(x_0,y_0,\ldots t_0)- \lambda&f_{xy}(x_0,y_0, \ldots t_0)&\cdots &f_{xt}(x_0,y_0,\ldots t_0)\\
        f_{yx}(x_0,y_0,\ldots t_0)&f_{yy}(x_0,y_0, \ldots t_0)- \lambda&\cdots &f_{yt}(x_0,y_0,\ldots t_0)\\
        \vdots &\vdots &\ddots &\vdots \\
        f_{tx}(x_0,y_0,\ldots t_0)&f_{ty}(x_0,y_0, \ldots t_0)&\cdots &f_{tt}(x_0,y_0,\ldots t_0)- \lambda\end{pmatrix}$

    \item \textbf{Auswertung:}
    
    \begin{tabular}{lll}
        \hline
        $\lambda_i < 0 \,\,\,\forall i$ &$\Longrightarrow$& $\text{lokales Maximum}$\\
        \hline
        $\lambda_i > 0 \,\,\,\forall i$ &$\Longrightarrow$& $\text{lokales Minimum}$\\
        \hline
        $\lambda_i > 0\,$ und $\,\lambda_i < 0$ &$\Longrightarrow$& $\text{Sattelpunkt}$\\
        \hline
    \end{tabular}

    \medskip
    Erklärung:
    \begin{itemize}
        \item $\lambda_i < 0 \,\,\,\forall i$ $\Leftrightarrow $ Alle $\lambda_i$ sind negativ
        \item $\lambda_i > 0 \,\,\,\forall i$ $\Leftrightarrow $ Alle $\lambda_i$ sind positiv
    \end{itemize}
\end{enumerate}


\subsection{Lokales oder Globales Extremum}
Für eine beliebige die Funktion $f(x, y, \ldots  , t)$ gilt:

$\boxed{\begin{array}{llll}
    f(x,y,\ldots ,t)\leq M_{\max}&\forall(x,y,\ldots ,t)\in\mathbb{D}_f&\Rightarrow&\text{globales Maxinum}\\
    f(x,y,\ldots ,t)>M_{\max}&\exists(x,y,\ldots ,t)\in\mathbb{D}_f&\Rightarrow&\text{kein globales Maximum}\\
    \hline f(x,y,\ldots ,t)\geq M_{\min}&\forall(x,y,\ldots ,t)\in\mathbb{D}_f&\Rightarrow&\text{globales Minimum}\\
    f(x,y,\ldots ,t)<M_{\min}&\exists(x,y,\ldots ,t)\in\mathbb{D}_f&\Rightarrow&\text{kein globales Minimum}
\end{array}}$

\medskip
\begin{tabular}{ll}
    $M_{\max}$: &\text{grösstes lokales Maximum}\\
    $M_{\min}$: &\text{kleinstes lokales Minimum}
\end{tabular}


        % \newpage
\section{Integration (bi-variat)}
Als bi-variate Integrale versteht man Integrale, die siech über zwei unabhängige Variablen erstrecken.
Sie haben die Form
\[
    \int\limits_{\Omega} f(\omega)\cdot \diff \omega =\int\limits_{X} \int\limits_{Y} f(x;y) \cdot dy \cdot dx
\]
wobei $ \Omega \subset \mathbb{R}^2 $, $ X \subset \mathbb{R} $ und $ Y \subset \mathbb{R} $ ist.

\subsection{Normalbereich}

TODO: WTF ist ein Normalbereich?
Schnitte sind Strecken (Intervalle) für x, y, ...

\subsection{Zweidimensionale Koordinatensysteme}
Neben den Kartesischen Koordinatensystemen kommen in zweidimensionalen Räumen auch Polare Koordinatensysteme zum Einsatz.
Die beiden Systeme können mit Hilfe der Trigonometrie in einander überführt werden.

\subsubsection{Umrechnung Kartesisch $\leftrightarrow$ Polar}
\begin{minipage}{0.29\linewidth}
    \myul{Polar zu Kartesisch}
    \[
    \begin{pmatrix}
        x \\
        y
    \end{pmatrix}
    =
    \begin{pmatrix}
        r * \cos{\varphi}\\
        r * \sin{\varphi}
    \end{pmatrix}
    \]
\end{minipage}
\hfill
\begin{minipage}{0.29\linewidth}
    \myul{Kartesisch zu Polar}
    \[
    \begin{pmatrix}
        r \\
        \varphi
    \end{pmatrix}
    =
    \begin{pmatrix}
        \sqrt{x^2+y^2}\\
        \tan^{-1}{\frac{y}{x}}
    \end{pmatrix}
    \]
\end{minipage}
\hfill
\begin{minipage}{0.29\linewidth}
    \begin{center}
        \begin{tikzpicture} [scale = 1.5]
            % Kartesische Achsen
            \draw[->] (0, 0) -- (1, 0) node [below] {$x$} ;
            \draw[->] (0, 0) -- (0, 1) node [left]  {$y$} ;

            % Punkt p
            \fill (0.5, 0.8) circle (1pt) node [anchor=south west] {$\vec{p}$};

            % Länge r
            \draw (0, 0) -- (0.5, 0.8) node [midway, above left] {$r$};
            % Winkel phi
            \draw [->] (0.5, 0) arc (0:58:0.5) node [midway, right] {$\varphi$};
        \end{tikzpicture}
    \end{center}
\end{minipage}

Dabei ist zu beachten, dass $\tan^{-1}$ nur werte von $-180\degree$ bis $180\degree$ liefert. 
$\varphi$ wird also, je nach dem in welchem Quadranten sich $\vec{p}$ befindet, nach folgendem Schema berechnet:
\begin{center}
    \begin{tikzpicture}
        % Achsen 
        \draw [->] (-2, 0) -- (2, 0) node [below] {$x$};
        \draw [->] (0, -0.8) -- (0, 0.8) node [left] {$y$};
        % Formeln
        \node at ( 1,  0.4) {$\tan^{-1}\frac{y}{x}$};
        \node at ( 1, -0.4) {$360\degree + \tan^{-1}\frac{y}{x}$};
        \node at (-1, -0.4) {$180\degree + \tan^{-1}\frac{y}{x}$};
        \node at (-1,  0.4) {$180\degree + \tan^{-1}\frac{y}{x}$};
    \end{tikzpicture}
\end{center}

Um eine ganzes Integral vom einen Koordinatensystem ins andere zu überführen, muss zum einen die Funktion $ f(x, y) $ zu $ f(r, \varphi) $ (oder umgekehrt) umgeschrieben, sowie die differentiale angepasst werden.
Hier dafür einige gängige Elemente:

\begin{tabular}{l c c}
                         & \bf{Kartesisch}                   & \bf{Polar}                                                       \\
    \bf{x-Achsenelement} & $\diff x$                         & $\diff x = \cos \varphi \diff r - r \sin \varphi \diff \varphi$  \\
    \bf{y-Achsenelement} & $\diff y$                         & $\diff x = \sin \varphi \diff r + r \cos \varphi \diff \varphi$  \\
    \bf{Linienelement  } & $\diff s^2 = \diff x^2 \diff y^2$ & $\diff s^2 = \diff r^2 + r^2 \diff \varphi^2$                    \\
    \bf{Flächenelement } & $\diff A = \diff x \diff y$       & $\diff A = r \diff r \diff \varphi$                              \\
\end{tabular}

\subsection{2D Transformation Polar zu Kartesisch}
TODO: Das isch ja ds gliiche wie obe beschribe, oder?
      Wänn da no meh ane sött wüsstich nöd was... -Flurin
T $=$ Transformation
\[
    \text{Polar } (r,\varphi) \xrightarrow{T} (x,y) \text{ Kartesisch}
\]

\[
\begin{pmatrix}
    x=r\cdot\cos(\varphi) \text{ } \cor{\mathbb{R}} \\
    y=r\cdot\sin(\varphi) \text{ } \cor{\mathbb{R}} 
\end{pmatrix}
\text{2D}
\]

Die Funktionen für $x$ und $y$ sind skalare Funktion.

    \begin{ctabular}{ll}
        $x=x(r;\varphi)$ & $ y=y(r;\varphi)$
    \end{ctabular}

\subsection{Derivative, Ableitung}
TODO: Idk was da ane söll -Flurin

\subsection{Anwendungsformeln Doppelintegral}
\resizebox{\linewidth}{!}{
    \begin{tabular}{|l|l|l|}
        \hline
        \bf{Allgemein} & \bf{Kartesische Koordinaten} & \bf{Polarkoordinaten} \\
        \hline

        \multicolumn{3}{|l|}{\bf{Flächeninhalt einer ebenen Figur}} \\\hline
        $ A = \int\limits_{A} \diff a $ & 
        $ = \int\limits_{X}\int\limits_{Y} \diff y \diff x $ &
        $ = \int\limits_{\Phi}\int\limits_{R} r \diff r \diff \varphi $ \\\hline

        \multicolumn{3}{|l|}{\bf{Oberfläche einer Ebene in drei Dimensionen}} \\\hline
        % TODO: Ka, ob das stimmt, evtl. umschreiben, dass man es besser versteht...
        % Ist aus Bernies unterlagen abgeschrieben.
        $ S = \int\limits_{A} \frac{1}{\cos \gamma} \diff a $ & 
        $ = \int\limits_{X}\int\limits_{Y} \sqrt{1 + \left(\frac{\partial z}{\partial x}\right)^2 + \left(\frac{\partial z}{\partial y}\right)^2} \diff y \diff x $ &
        $ = \int\limits_{\Phi}\int\limits_{R} \sqrt{r^2 + r^2\left(\frac{\partial z}{\partial r}\right)^2 + \left(\frac{\partial z}{\partial \varphi}\right)^2} \diff r \diff \varphi $ \\\hline
        
        \multicolumn{3}{|l|}{\bf{Volumen eines Zylinders}} \\\hline
        $ V = \int\limits_{A} z \diff a $ & 
        $ = \int\limits_{X}\int\limits_{Y} z \diff y \diff x $ &
        $ = \int\limits_{\Phi}\int\limits_{R} z r \diff r \diff \varphi $ \\\hline

        \multicolumn{3}{|l|}{\bf{Trägheitsmoment einer ebenen Figur, bezogen auf die x-Achse}} \\\hline
        $ I_x = \int\limits_{A} y^2 \diff a $ & 
        $ = \int\limits_{X}\int\limits_{Y} y^2 \diff y \diff x $ &
        $ = \int\limits_{\Phi}\int\limits_{R} r^3 \sin^2\varphi \diff r \diff \varphi $ \\\hline

        \multicolumn{3}{|l|}{\bf{Trägheitsmoment einer ebenen Figur, bezogen auf den Pol $(0, 0)$}} \\\hline
        % TODO: wieso r^3? evtl. nachrechnen, scheint iwie komisch...
        $ I_x = \int\limits_{A} r^2 \diff a $ & 
        $ = \int\limits_{X}\int\limits_{Y} (x^2 + y^2) \diff y \diff x $ &
        $ = \int\limits_{\Phi}\int\limits_{R} r^3 \diff r \diff \varphi $ \\\hline

        \multicolumn{3}{|l|}{\bf{Masse einer ebenen Figur mit Dichtefunktion $\varrho$}} \\\hline
        $ m = \int\limits_{A} \varrho \diff a $ & 
        $ = \int\limits_{X}\int\limits_{Y} \varrho (x, y) \diff y \diff x $ &
        $ = \int\limits_{\Phi}\int\limits_{R} \varrho (x, y) r \diff r \diff \varphi $ \\\hline

        \multicolumn{3}{|l|}{\bf{Koordinaten des Schwerpunkts einer homogenen, ebenen Figur}} \\\hline
        $ x_{COG} = \frac{\int\limits_{A} x \diff a}{A} $ & 
        $ = \frac{\int\limits_{X}\int\limits_{Y} x \diff y \diff x}{\int\limits_{X}\int\limits_{Y} \diff y \diff x} $ &
        $ = \frac{\int\limits_{\Phi}\int\limits_{R} r^2 \cos \varphi \diff r \diff \varphi}{\int\limits_{\Phi}\int\limits_{R} r \diff r \diff \varphi} $ \\
        $ y_{COG} = \frac{\int\limits_{A} y \diff a}{A} $ & 
        $ = \frac{\int\limits_{X}\int\limits_{Y} y \diff y \diff x}{\int\limits_{X}\int\limits_{Y} \diff y \diff x} $ &
        $ = \frac{\int\limits_{\Phi}\int\limits_{R} r^2 \sin \varphi \diff r \diff \varphi}{\int\limits_{\Phi}\int\limits_{R} r \diff r \diff \varphi} $ \\\hline
    \end{tabular}
}
\smallskip

        % TODO: Koordinatentransformation (Längenverzerrung, Elementverzerrung, Längenelement)
        % TODO: Fubini (Integration)
        % TODO: Transformation / Jacobi
        % \newpage
\section{Integration (multi-variat)}
% TODO: Koordinatentransformation (Längenverzerrung, Elementverzerrung, Längenelement)
\subsection{Dreidimensionale Koordinatensysteme}
\resizebox{\linewidth}{!}{
    \begin{tabular}{c c c}
        \myul{\textbf{Kartesisch}} & \myul{\textbf{Zylindrisch}} & \myul{\textbf{Kubisch}} \\
        
        \tdplotsetmaincoords{70}{110}
        \begin{tikzpicture}[tdplot_main_coords, scale=2]
            % Koordinatensystem
            \draw [-{latex}] (0, 0, 0) -- (1, 0, 0) node [below left] {$x$};
            \draw [-{latex}] (0, 0, 0) -- (0, 1, 0) node [right]      {$y$};
            \draw [-{latex}] (0, 0, 0) -- (0, 0, 1) node [right]      {$z$};
            % Punkt bei (0.75,0.75,0.75)
            \fill (0.75, 0.75, 0.75) circle (1pt) node [above right] {$P(x, y, z)$};
            % Koordinatenkomponenten
            \draw [dashed] (0, 0.75, 0)    -- (0.75, 0.75, 0)    node [midway, below right] {$x$};
            \draw [dashed] (0.75, 0, 0)    -- (0.75, 0.75, 0)    node [midway, below left]  {$y$};
            \draw [dashed] (0.75, 0.75, 0) -- (0.75, 0.75, 0.75) node [midway, right]       {$z$};
        \end{tikzpicture} &

        \tdplotsetmaincoords{70}{110}
        \begin{tikzpicture}[tdplot_main_coords, scale=2]
            % Koordinatensystem
            \draw [-{latex}] (0, 0, 0) -- (1, 0, 0) node [below left] {$x$};
            \draw [-{latex}] (0, 0, 0) -- (0, 1, 0) node [right]      {$y$};
            \draw [-{latex}] (0, 0, 0) -- (0, 0, 1) node [right]      {$z$};
            % Punkt bei (0.75,0.75,0.75)
            \fill (0.75, 0.75, 0.75) circle (1pt) node [above right] {$P(r, \phi, z)$};
            % Koordinatenkomponenten
            \draw [dashed] (0,0,0)         --  (0.75, 0.75, 0)    node [midway, above right] {$r$};
            \draw [-{latex}]     (0.5,0,0)       arc (0:45:0.5)         node [midway, below]       {$\phi$};
            \draw [dashed] (0.75, 0.75, 0) --  (0.75, 0.75, 0.75) node [midway, right]       {$z$};
        \end{tikzpicture} &

        \tdplotsetmaincoords{70}{110}
        \begin{tikzpicture}[tdplot_main_coords, scale=2]
            % Koordinatensystem
            \draw [-{latex}] (0, 0, 0) -- (1, 0, 0) node [below left] {$x$};
            \draw [-{latex}] (0, 0, 0) -- (0, 1, 0) node [right]      {$y$};
            \draw [-{latex}] (0, 0, 0) -- (0, 0, 1) node [right]      {$z$};
            % Punkt bei (0.75,0.75,0.75)
            \fill (0.75, 0.75, 0.75) circle (1pt) node [above right] {$P(r, \theta, \phi)$};
            % Koordinatenkomponenten
            \draw [dotted] (0,0,0) -- (0.75, 0.75, 0) -- (0.75, 0.75, 0.75);
            \draw [-{latex}]     (0.5,0,0) arc (0:45:0.5)         node [midway, below]       {$\phi$};
            \draw [dashed] (0, 0, 0) --  (0.75, 0.75, 0.75) node [midway, below right] {$r$};
            \tdplotsetthetaplanecoords {90}
            \tdplotdrawarc [tdplot_rotated_coords, -{latex}] {(0, 0, 0)} {0.5} {0} {45} {anchor=south} {$\theta$}
        \end{tikzpicture} \\

        $ 
        \begin{pmatrix}
            x \\ y \\ z
        \end{pmatrix} 
        =
        \begin{pmatrix}
            r \cos \phi \\ r \sin \phi \\ z
        \end{pmatrix}
        =
        \begin{pmatrix}
            r \sin \theta \cos \phi \\ r \sin \theta \sin \phi \\ r \cos \theta
        \end{pmatrix}
        $ &
        $ 
        \begin{pmatrix}
            r_{zyl} \\ \phi \\ z
        \end{pmatrix} 
        =
        \begin{pmatrix}
            \sqrt{x^2 + y^2} \\ \tan^{-1}\frac{y}{x} \\ z
        \end{pmatrix}
        =
        \begin{pmatrix}
            r_{sph} \sin \theta \\ \phi \\ r_{sph} \cos \theta
        \end{pmatrix}
        $ &
        $ 
        \begin{pmatrix}
            r_{sph} \\ \theta \\ \phi
        \end{pmatrix} 
        =
        \begin{pmatrix}
            \sqrt{x^2+y^2+z^2} \\ \cos^{-1} \frac{z}{r_{sph}} \\ \sgn(y) \cos^{-1} \frac{x}{\sqrt{x^2+y^2}}
        \end{pmatrix}
        =
        \begin{pmatrix}
            \sqrt{r_{zyl}^2+z^2} \\ \tan^{-1}\frac{r_{zyl}}{z} \\ \phi
        \end{pmatrix}
        $ \\
    \end{tabular}
}
\smallskip
\subsubsection{Umrechnen zwischen Koordinatensystemen}
Beim Umrechnen zwischen den Koordinatensystemen gelten im Grunde genommen die obigen Formeln. 
Dabei muss jedoch in einigen Fällen auf die Wertebereiche von den trigonometrischen Funktionen rücksicht genommen werden.

\myul{\textbf{Zylindrisch $\rightarrow$ Kartesisch:}}\\
\myul{\textbf{Sphärisch $\rightarrow$ Kartesisch:}}\\
Keine weiteren Berücksichtigungen nötig, die Berechnung erfolgt nach der Formel oben.
\medskip

\myul{\textbf{Kartesisch $\rightarrow$ Zylindrisch:}}\\
\begin{minipage}{0.49\linewidth}
    Der Parameter $\phi$ wird analog zum zweidimensionalen Fall, je nach dem in welchem Quadranten sich $P$ befindet, nach dem Schema rechts berechnet.
\end{minipage}
\hfill
\begin{minipage}{0.49\linewidth}
    \begin{center}
        \begin{tikzpicture}
            % Achsen 
            \draw [-{latex}] (-2, 0) -- (2, 0) node [below] {$x$};
            \draw [-{latex}] (0, -0.8) -- (0, 0.8) node [left] {$y$};
            % Formeln
            \node at ( 1,  0.4) {$       \tan^{-1}\frac{y}{x}$};
            \node at ( 1, -0.4) {$2\pi + \tan^{-1}\frac{y}{x}$};
            \node at (-1, -0.4) {$ \pi + \tan^{-1}\frac{y}{x}$};
            \node at (-1,  0.4) {$ \pi + \tan^{-1}\frac{y}{x}$};
        \end{tikzpicture}
    \end{center}
\end{minipage}
\medskip

\myul{\textbf{Sphärisch $\rightarrow$ Zylindrisch:}}\\
\myul{\textbf{Kartesisch $\rightarrow$ Sphärisch:}}\\
Keine weiteren Berücksichtigungen nötig, die Berechnung erfolgt nach der Formel oben.
\medskip

\myul{\textbf{Zylindrisch $\rightarrow$ Sphärisch:}}\\
\begin{minipage}{0.49\linewidth}
    Auch hier macht der $\tan^{-1}$ Probleme, da er Werte von $-\frac{\pi}{2}$ bis $\frac{\pi}{2}$ liefert, für $\theta$ jedoch $\theta \in [0, \pi]$ gelten soll.
    Je nach dem, ob $P$ sich oberhalb oder unterhalb der $xy$-Ebene befindet, wird $\theta$ wie rechts berechnet.
\end{minipage}
\hfill
\begin{minipage}{0.49\linewidth}
    \begin{center}
        \begin{tikzpicture}
            % Achsen 
            \draw [-{latex}] (-2,  0)   -- (2, 0) node [below] {$xy$-Ebene};
            \draw [-{latex}] ( 0, -0.8) -- (0,.9) node [left]  {$z$};
            % Formeln
            \node [fill=white] at (0,  0.4) {$\tan^{-1}\frac{r_{zyl}}{z}$};
            \node [fill=white] at (0, -0.4) {$\pi + \tan^{-1}\frac{r_{zyl}}{z}$};
        \end{tikzpicture}
    \end{center}
\end{minipage}

\subsection{Längenintegrale}
\subsubsection{Längenelemente}\label{section:int_multivar:längenelemente}
$$
 \diff s^2 
    = \underbrace{\diff x^2 + \diff y^2 + \diff z^2}_{\text{Kartesisch}}
    = \underbrace{\diff r^2 + r^2 \diff \theta^2 + \diff z^2}_{\text{Zylindrisch}}
    = \underbrace{\diff r^2 + r^2 \diff \theta^2 + r^2 \sin^2 \theta \diff \phi^2}_{\text{Sphärisch}}
$$
\subsubsection{Länge einer Funktion}
Die Bestimmung der Länge einer Kurve kann in folgende Schritte unterteilt werden:
\begin{enumerate}
    \item \myul{\textbf{Funktion in die Parameterdarstellung überführen (sofern nicht gegeben):}}
    \item[] Dafür wird einer der Parameter (z.B. $x$ oder $\theta$) $=t$ gesetzt und die anderen Parameter ebenfals als Funktion von $t$ ausgedrückt.
    \item \myul{\textbf{Integral aufstellen:}}
    \item[] Das Integral in der Form $ \iiint ds $ wird mit $\frac{\diff t}{\diff t}$ erweitert.
    \item \myul{\textbf{Das Integral lösen}}
\end{enumerate}

\subsubsection{Beispiel}
Es soll die Länge der Kurve $\vec{v}(t) = \begin{pmatrix}x(t)\\y(t)\\z(t)\end{pmatrix}$ auf dem Interval $[t_1, t_2]$ bestimmt werden.
Dazu werden die oben genannten Schritte abgearbeitet:
\begin{enumerate}
    \item \textbf{Funktion in die Parameterdarstellung überführen}
    \item[] Hier nicht nötig. % evtl. TODO: Beispiel wählen, bei dem das nötig ist.
    \item \textbf{Integral aufstellen} 
    \item[] $ \iiint ds = \iiint \sqrt{\diff x^2 + \diff y^2 + \diff z^2} = \int_{t_1}^{t_2} \sqrt{\left(\frac{\diff x}{\diff t}\right)^2 + \left(\frac{\diff y}{\diff t}\right)^2 + \left(\frac{\diff z}{\diff t}\right)^2} \diff t$
    \item Integral lösen
    \item[] $\frac{\diff x}{\diff t}$, $\frac{\diff y}{\diff t}$ und $\frac{\diff z}{\diff t}$ ausrechnen, einsetzen, integrieren.
\end{enumerate}

\subsection{Flächenintegrale}
\subsubsection{Flächenelemente}
% TODO: stimmt das?
Das Bestimmen der Flächenelemente ist in drei Dimensionen nicht wie bei den Längen- und Volumenelementen pauschal möglich.
Dies, da jeweils nur über zwei der drei Koordinaten integriert werden muss.
Ein einfaches Verfahren für das Berechnen von Flächeninhalten schafft jedoch abhilfe.
\subsubsection{Flächeninhalt einer Oberfläche}
Für das Berechnen der Oberflächen von Funktionen des Typs $f(a, b)$ in 3D kann die Formel
$$ S = \int\limits_{B} \int\limits_{A} \sqrt{(f_{a})^2 + (f_{b})^2 + 1} \diff a \diff b $$
verwendet werden. Dabei repräsentieren $a$ und $b$ die beiden Koordinatenrichtungen, in denen sich die Fläche erstreckt.
$f_a$ und $f_b$ sind die partiellen Ableitungen der Funktion $f(a, b)$ nach $a$ bzw. $b$.
\medskip

\myul{\bf{Beispiele zur Veranschaulichung:}}\\
Es soll die Oberfläche der Funktion $ f(x, y) $ im Bereich $ x \in [x_1, x_2], y \in [y_1, y-2] $ bestimmt werden.
Das entsprechende integral lautet:
$$ S = \int_{y_1}^{y_2} \int_{x_1}^{x_2} \sqrt{(f_{x})^2 + (f_{y})^2 + 1} \diff x \diff y $$

Wäre die Funktion $f$ stat in kartesischen in polaren oder sphärischen Koordinaten formuliert, ändern sich lediglich die Namen der Variablen. 
Folglich ist das zu einer in sphärischen Koordinaten definierten Fkt. $f(\theta, \phi)$ gehörende Integral
$$ S = \int_{\phi_1}^{\phi_2} \int_{\theta_1}^{\theta_2} \sqrt{(f_{\theta})^2 + (f_{\phi})^2 + 1} \diff \theta \diff \phi $$
sehr leicht aufzustellen.

\subsection{Volumenintegrale}
\subsubsection{Volumenelemente}
$$ 
 \diff V 
    = \underbrace{\diff x \diff y \diff z}_{\text{Kartesisch}}
    = \underbrace{r \diff r \diff \phi \diff z}_{\text{Zylindrisch}}
    = \underbrace{r^2 \sin \theta \diff \theta \diff \phi \diff r}_{\text{Sphärisch}}
$$

\subsection{Anwendungen Trippel-Integrale}
\resizebox{\linewidth}{!}{
    \begin{tabular}{|l|l|l|l|}  
        \hline
        \bf{Allgemein} & \bf{Kartesische Koordinaten} & \bf{Zylinderkoordinaten} & \bf{Kugelkoordinaten} \\
        \hline
        \multicolumn{4}{|l|}{\bf{Volumen eines Körpers $K$}} \\
        \hline
        $ V = \iiint\limits_{K} \diff V $ & 
        $ = \iiint \diff x \diff y \diff z $ &
        $ = \iiint r \diff r \diff \phi \diff z $ &
        $ = \iiint r^2 \sin \theta \diff \theta \diff \phi \diff r $ \\
        \hline

        \multicolumn{4}{|l|}{\bf{Trägheitsmoment eines Körpers $K$, bezogen auf die Z-Achse}} \\
        \hline
        $ I_z = \iiint\limits_{K} r^2 \diff V $ & 
        $ = \iiint (x^2 + y^2) \diff x \diff y \diff z $ &
        $ = \iiint (r^2) r \diff r \diff \phi \diff z $ &
        $ = \iiint (r^2 \sin^2 \theta) r^2 sin \theta \diff \theta \diff \phi \diff r $ \\
        \hline

        \multicolumn{4}{|l|}{\bf{Masse eines Körpers $K$ mit der Dichtefunktion $\varrho$}} \\
        \hline
        $ M = \iiint\limits_{K} \varrho \diff V $ & 
        $ = \iiint \varrho (x, y, z) \diff x \diff y \diff z $ &
        $ = \iiint \varrho (r, \phi, z) r \diff r \diff \phi \diff z $ &
        $ = \iiint \varrho (r, \theta, \phi) r^2 sin \theta \diff \theta \diff \phi \diff r $ \\
        \hline

        \multicolumn{4}{|l|}{\bf{Koordinaten des Schwerpunktes $S$ eines homogenen Körpers $K$}} \\
        \hline
        $ x_{S} = \frac{\iiint\limits_{K} x \diff V}{V} $ & 
        $ = \frac{\iiint (x) \diff x \diff y \diff z}{V} $ &
        $ = \frac{\iiint (r \cos \phi) r \diff r \diff \phi \diff z}{V} $ &
        $ = \frac{\iiint (r \sin \theta \cos \phi) r^2 sin \theta \diff \theta \diff \phi \diff r}{V} $ \\

        $ y_{S} = \frac{\iiint\limits_{K} y \diff V}{V} $ & 
        $ = \frac{\iiint (y) \diff x \diff y \diff z}{V} $ &
        $ = \frac{\iiint (r \sin \phi) r \diff r \diff \phi \diff z}{V} $ &
        $ = \frac{\iiint (r \sin \theta \sin \phi) r^2 sin \theta \diff \theta \diff \phi \diff r}{V} $ \\

        $ z_{S} = \frac{\iiint\limits_{K} z \diff V}{V} $ & 
        $ = \frac{\iiint (z) \diff x \diff y \diff z}{V} $ &
        $ = \frac{\iiint (z) r \diff r \diff \phi \diff z}{V} $ &
        $ = \frac{\iiint (r \cos \theta) r^2 sin \theta \diff \theta \diff \phi \diff r}{V} $ \\
        \hline
    \end{tabular}
}
\smallskip
Hinweis: Damit die Volumenelemente leichter erkennbar und die Formeln entsprechend besser nachvollziebar sind, wurden sie teilweise nicht vollständig vereinfacht.

        % TODO: Fubini (Integration)
        % TODO: Transformation / Jacobi
        % \section{Differenziation und Integration von Kurven}


        % TODO: Kurvenintegral 2. Art
        % \input{sections/08_ober-flächenintegrale.tex}
        % TODO: Allgemeine Wendelfläche
        % TODO: Freie Fläche (parametrisiert)
        % TODO: 1. metrischer Tensor
        
        % TODO: REMOVE FOLLOWING LINE FOR FINAL VERSION
        \setcounter{section}{8}
        \section{Vektoranalysis}


        % TODO: Koordinatensysteme (Kartesisch, Polar, Kugel (Geografisch & Math.))
        % TODO: Vektorfelder
        % TODO: Skalarer Durchfluss
        % TODO: Spezialfälle
        % TODO: Vektor Längenelement
        % TODO: Divergenz (Volumenableitung) (Cartesisch; Nabla / del-Operator)
        % TODO: Poisson-Gleichung (Laplace-Gleichung)
        % TODO: rot()
        % TODO: Stokes Integralsatz
    \end{layout}
\end{document}
