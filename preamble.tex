% =========================================== PACKAGES ============================================
\usepackage[utf8]{inputenc}         % input encoding: UTF-8
\usepackage[T1]{fontenc}            % font encoding: T1
\usepackage{textcomp}               % additional symbols
\usepackage{times}                  % times new roman font
\usepackage[main=ngerman]{babel}    % set main language to german


\usepackage{multicol}               % provides multicols environment
\usepackage{geometry}               % set page layout


\usepackage{enumitem}               % list customization
\usepackage{outlines}               % easy nested lists
\usepackage{tabularx}               % some nicer tables with X columns
\usepackage{multirow}               % multirow in tables
\usepackage{hhline}                 % double lines in tables
\usepackage{array}
\usepackage{makecell}               % multi-line cells in tables
\usepackage{colortbl}               % colored tables
\usepackage{booktabs}               % different table rules


\usepackage{amsmath}                % math symbols
\usepackage{amssymb}                % more math symbols
\usepackage{mathtools}
\usepackage{mhsetup}
\usepackage{txfonts}
\usepackage[squaren]{SIunits}       % SI-units
\usepackage{bm}                     % bold math symbols
\usepackage{trfsigns}               % needed for "Laplace" symbol (Korrespondenz)
\usepackage{mathrsfs}               % needed for Fourier transform "F"


\usepackage{graphicx}               % include graphics
\usepackage{graphbox}               % needed for aligning images in multicol environment
\usepackage{scalerel}               % scale any objects
\usepackage{anyfontsize}            % set any font size
\usepackage{xcolor}                 % needed for colors


\usepackage{tcolorbox}              % colored boxes
\usepackage[outline]{contour}       % contour for text (used in custom underline command)
\usepackage[normalem]{ulem}         % custom underline (used in custom underline command)


\usepackage{tikz}                   % needed for TikZ drawings


% \usepackage{listings}               % for nicer code display
% to use nodes inside listing see: https://texample.net/tikz/examples/tikz-listings/


\usepackage{hyperref}               % clickable links
\usepackage{qrcode}                 % QR code generation (also clickable)


\usepackage{ifthen}                 % if-then-else commands
\usepackage{calc}                   % simple arithmetic in LaTeX commands


\usepackage{draftwatermark}         % watermark on pages after a certain limit
\usepackage{fancyhdr}               % custom header and footer
\usepackage[explicit]{titlesec}     % custom section titles


\usepackage{datetime2}              % custom date format for versioning


% ========================================== BASIC SETUP ==========================================

% --------------------------------------- DOCUMENT SETTINGS ---------------------------------------
\hypersetup{hidelinks,
% set pdf metadata
            pdfauthor={\author},
            pdftitle={\shorttitle},
            pdfsubject={\title\ \semester},
            pdfkeywords={Gahn go lerne!!}}

% set style for URLs
\urlstyle{same} % sets url font to the same as the preceeding text

% set page layout
\geometry{left=3mm, 
          right=3mm, 
          top=3mm, 
          bottom=6mm, 
          headheight=0mm, 
          headsep=0mm, 
          footskip=4mm}

\setlength{\columnsep}{1.5mm}       % distance between columns
\setlength{\columnseprule}{0.1pt}   % thickness of column separation line
\setlength{\parindent}{0pt}         % no paragraph indentation

\setcounter{tocdepth}{2}            % only display sections and subsections in toc
% \setcounter{secnumdepth}{0}       % uncomment to disable section numbering

\DeclareMathSizes{8}{8}{6}{4}       % set math font sizes for 8pt document


% --------------------------------------- COLOR DEFINITIONS ---------------------------------------
\definecolor{sectioncolor}{HTML}{8acafc}
\definecolor{subsectioncolor}{HTML}{c3e3fd}
\definecolor{sectextcol}{HTML}{000000}
\definecolor{subsectextcol}{HTML}{000000}

\definecolor{backcolour}{HTML}{f5f5f0} % background color for highlighted text

% TODO: define color palette
% color palette: https://colorkit.co/color-palette-generator/FF8552-9e22bd-404E7C-C32E15-225A28/
\definecolor{green}{HTML}{1af430}
\definecolor{red}{HTML}{f90d09}
\definecolor{blue}{HTML}{093ce5}
\definecolor{orange}{HTML}{f7730e}
\definecolor{violet}{HTML}{a516c9}

% colors for listings (code)
% \definecolor{commentcolour}{HTML}{404E7C}
% \definecolor{keywordcolour}{HTML}{225A28}
% \definecolor{stringcolour}{HTML}{9e22bd}
% \definecolor{numbercolour}{HTML}{808080}


% ----------------------------------- LIST AND TABULAR SETTINGS -----------------------------------
\setlist[enumerate]{label=\bfseries\arabic*.,   % label style bold arabic numerals (1., 2., ...)
                    leftmargin=*}               % remove left margin from enumerate
\setlist[itemize]{leftmargin=1.5em}             % left margin for itemize: 1.5em
\setlist{nosep}                                 % no vertical spacing between list items

\renewcommand{\arraystretch}{1.2}               % stretch table rows


% ----------------------------------------- TIKZ SETTINGS -----------------------------------------
\usetikzlibrary{arrows}
\usetikzlibrary{arrows.meta}
\usetikzlibrary{bending}
\usetikzlibrary{decorations.pathreplacing}
\usetikzlibrary{angles}
\usetikzlibrary{tikzmark}
\usetikzlibrary{petri}
\usetikzlibrary{positioning}
\usetikzlibrary{shapes}
\usetikzlibrary{calc}
\usetikzlibrary{fit}


% ------------------------------------ OTHER PACKAGE SETTINGS -------------------------------------

% define and set new date style for versioning as YYYYMMDD
\DTMnewdatestyle{vnumdate}{%
    \renewcommand{\DTMdisplaydate}[4]{\number##1\DTMtwodigits{##2}\DTMtwodigits{##3}}%
    \renewcommand{\DTMDisplaydate}{\DTMdisplaydate}%
}
\DTMsetdatestyle{vnumdate}


% setup for ulem and contour packages
\renewcommand{\ULdepth}{1.75pt} % set underline depth
\contourlength{0.7pt}           % set contour length


% ====================================== SETUP AND COMMANDS =======================================

% custom font size for paragraph titles
\newcommand{\semilarge}{\fontsize{9}{8}\selectfont} % new font size \semilarge (9pt)


% custom underline command for exclusions on lowercase letters such as g, j, p, q, y
\newcommand{\myul}[1]{%
    \uline{\phantom{#1}}%
    \llap{\contour*{white}{#1}}%
}


% setup header and footer
\pagestyle{fancy}
\fancyhf{}                          % clear all header and footer fields
\renewcommand{\headrulewidth}{0pt}  % remove header rule
\renewcommand{\footrulewidth}{0pt}  % remove footer rule
\fancyfoot[C]{\thepage}             % page number in center of footer


% --------------------------------------- TITLE FORMATTING ----------------------------------------

% section formatting
\titleformat{\section}
            % {\fontsize{9}{8}\selectfont\bfseries}
            {\Large\bfseries}
            {}
            {0mm}
            {\tikz{
                \node[fill=sectioncolor,            % fill color:       sectioncolor
                      text=sectextcol,              % text color:       sectextcol
                      text width=\columnwidth-4pt,  % text width:       columnwidth - 2x padding
                      text depth=0pt,               % text depth:       0pt (needed so text stays vertically centered)
                      minimum height=5mm,           % minimum height:   5mm
                      inner sep=2pt,                % inner padding:    2pt
                      align=left]                   % text alignment:   left
                      {\thesection\ #1};}}

\titleformat{numberless, name=\section}
            % {\fontsize{9}{8}\selectfont\bfseries}
            {\Large\bfseries}
            {}
            {0mm}
            {\tikz{
                \node[fill=sectioncolor,            % fill color:       sectioncolor
                      text=sectextcol,              % text color:       sectextcol
                      text width=\columnwidth-4pt,  % text width:       columnwidth - 2x padding
                      text depth=0pt,               % text depth:       0pt (needed so text stays vertically centered)
                      minimum height=5mm,           % minimum height:   5mm
                      inner sep=2pt,                % inner padding:    2pt
                      align=left]                   % text alignment:   left
                      {#1};}}

\titlespacing{\section}
             {0mm}
             {.8ex}
             {.8ex}


% subsection formatting
\titleformat{\subsection}
            {\large\bfseries}
            {}
            {0mm}
            {\phantomsection\tikz{
                \node[fill=subsectioncolor,         % fill color:       subsectioncolor 
                      text=subsectextcol,           % text color:       subsectextcol 
                      text width=\columnwidth-4pt,  % text width:       columnwidth - 2x padding 
                      text depth=0pt,               % text depth:       0pt (needed so text stays vertically centered)
                      minimum height=5mm,           % minimum height:   5mm 
                      inner sep=2pt,                % inner padding:    2pt 
                      align=left]                   % text alignment:   left
                      {\thesubsection\ #1};}}

\titleformat{numberless, name=\subsection}
            {\large\bfseries}
            {}
            {0mm}
            {\phantomsection\tikz{
                \node[fill=subsectioncolor,         % fill color:       subsectioncolor 
                      text=subsectextcol,           % text color:       subsectextcol 
                      text width=\columnwidth-4pt,  % text width:       columnwidth - 2x padding 
                      minimum height=5mm,           % minimum height:   5mm 
                      inner sep=2pt,                % inner padding:    2pt 
                      align=left]                   % text alignment:   left
                      {#1};}}

\titlespacing{\subsection}
             {0mm}
             {.8ex}
             {.8ex}


% subsubsection formatting
\titleformat{\subsubsection}
            % {\fontsize{9}{8}\selectfont\bfseries}
            {\large\bfseries}
            {\myul{\thesubsubsection\ }}
            {0mm}
            {\phantomsection{}\myul{#1}}

\titlespacing{\subsubsection}
             {0mm}
             {1ex}
             {1ex}


% paragraph formatting
\titleformat{\paragraph}[runin]
            {\semilarge\bfseries}
            {}
            {0mm}
            {#1\normalfont :\;}

\titlespacing{\paragraph}
             {0mm}
             {0ex}
             {0.1ex}


% new alias for paragraph '\para' (shorter than \paragraph)
\let\para\paragraph%

% custom command for examples
\newcommand{\example}[1]{\subsubsection*{Beispiel: #1}}


% ----------------------------------- CUSTOM TABULAR SPECIFIERS -----------------------------------

% centered fixed width column type
\newcolumntype{P}[1]{>{\centering\arraybackslash}p{#1}}

 % centered variable width column type
\newcolumntype{C}{>{\centering\arraybackslash}X}

% centered math column type
\newcolumntype{M}{>{$}c<{$}}

% right aligned math column type
\newcolumntype{O}{>{$}r<{$}}


% inline tikz node for later referencing
\newcommand{\tikznode}[2]{% from https://tex.stackexchange.com/a/402466/121799
	\ifmmode%
	\tikz[remember picture,baseline= (#1.base),inner sep=0pt] \node(#1){$#2$};
	\else
	\tikz[remember picture,baseline= (#1.base),inner sep=0pt] \node(#1){#2};
	\fi}


% custom inline tcolorbox
\newtcbox{\mybox}
            [1]
            [backcolour]
            {on line,
            arc=0pt,
            outer arc=0pt,
            colback=#1,
            colframe=#1,
            boxsep=0pt,
            left=1pt,
            right=1pt,
            top=1pt,
            bottom=1pt,
            boxrule=0pt}


\makeatletter

% ------------------------------- SECTIONING COMMANDS CUSTOMIZATION -------------------------------


% section: add optional argument to command for script page numbers
\let\old@sec\section%
\RenewDocumentCommand{\section}{somg}{%
    \IfBooleanTF{#1}{
        \IfNoValueTF{#2}{
            \IfNoValueTF{#4}{
                \old@sec*{#3}
            }{
                \old@sec*{#3 {\small(S. #4)}}
            }
        }{
            \IfNoValueTF{#4}{
                \old@sec*[#2]{#3}
            }{
                \old@sec*[#2]{#3 {\small(S. #4)}}
            }
        }%
    }{
        \IfNoValueTF{#2}{
            \IfNoValueTF{#4}{
                \old@sec{#3}
            }{
                \old@sec{#3 {\small(S. #4)}}
            }
        }{
            \IfNoValueTF{#4}{
                \old@sec[#2]{#3}
            }{
                \old@sec[#2]{#3 {\small(S. #4)}}
            }
        }%
    }
}


% subsection: add optional argument to command for script page numbers
\let\old@subsec\subsection%
\RenewDocumentCommand{\subsection}{somg}{%
    \IfBooleanTF{#1}{
        \IfNoValueTF{#2}{
            \IfNoValueTF{#4}{
                \old@subsec*{#3}
            }{
                \old@subsec*{#3 {\small(S. #4)}}
            }
        }{
            \IfNoValueTF{#4}{
                \old@subsec*[#2]{#3}
            }{
                \old@subsec*[#2]{#3 {\small(S. #4)}}
            }
        }%
    }{
        \IfNoValueTF{#2}{
            \IfNoValueTF{#4}{
                \old@subsec{#3}
            }{
                \old@subsec{#3 {\small(S. #4)}}
            }
        }{
            \IfNoValueTF{#4}{
                \old@subsec[#2]{#3}
            }{
                \old@subsec[#2]{#3 {\small(S. #4)}}
            }
        }%
    }
}


% subsubsection: add optional argument to command for script page numbers
\let\old@subsubsec\subsubsection%
\RenewDocumentCommand{\subsubsection}{somg}{%
    \IfBooleanTF{#1}{
        \IfNoValueTF{#2}{
            \IfNoValueTF{#4}{
                \old@subsubsec*{#3}
            }{
                \old@subsubsec*{#3 {\small(S. #4)}}
            }
        }{
            \IfNoValueTF{#4}{
                \old@subsubsec*[#2]{#3}
            }{
                \old@subsubsec*[#2]{#3 {\small(S. #4)}}
            }
        }%
    }{
        \IfNoValueTF{#2}{
            \IfNoValueTF{#4}{
                \old@subsubsec{#3}
            }{
                \old@subsubsec{#3 {\small(S. #4)}}
            }
        }{
            \IfNoValueTF{#4}{
                \old@subsubsec[#2]{#3}
            }{
                \old@subsubsec[#2]{#3 {\small(S. #4)}}
            }
        }%
    }
}


% custom text rightarrow to match tikz arrows
\renewcommand{\textrightarrow}{
    \tikz{
        \draw[-{Stealth[length=1.7mm]},
              double]
                (0,0) to (0.3,0);}}

% custom text leftrightarrow to match tikz arrows
\newcommand{\textlrarrow}{
    \tikz{
        \draw[{Stealth[length=1.7mm]}-{Stealth[length=1.7mm]},
              double]
                (0,0) to (0.4,0);}}


% renews the pmatrix environment to use \lgroup and \rgroup instead of \left( and \right)
\renewenvironment{pmatrix}{%
    \left\lgroup%
    \matrix@check\pmatrix\env@matrix%
}{
    \endmatrix\right\rgroup%
}

% renews the pmatrix* environment to use \lgroup and \rgroup instead of \left( and \right)
\MHInternalSyntaxOn%
\renewenvironment{pmatrix*}[1][c]
  {\left\lgroup\MT_matrix_begin:N #1}
  {\MT_matrix_end:\right\rgroup}
\MHInternalSyntaxOff%


% custom command for size matched colored brackets
\newcommand{\bbr}[2]{\colorlet{saved}{.}\color{#1}\left\lgroup\color{saved}#2\color{#1}\right\rgroup\color{saved}}

% custom command for colore underbrace
\newcommand{\cub}[3]{\colorlet{saved}{.}\color{#1}\ensuremath{\underbrace{\color{saved}#2}_{\color{#1}#3}\color{saved}}}

% custom command for differential operator d
\newcommand{\diff}{\ensuremath{\mathop{} \! \mathrm{d}}}

\newcommand{\rreal}{\ensuremath{\mathbb{R}}}
\newcommand{\nnatural}{\ensuremath{\mathbb{N}}}

% custom command for underset limes operator
\newcommand{\limes}[1]{\ensuremath{\underset{#1}{\lim}}}

% custom command for absolute value
\newcommand{\abs}[1]{\ensuremath{\left|#1\right|}}

% custom command for gradient function
\newcommand{\grad}{\ensuremath{\operatorname{grad}}}

% custom command for divergence operator -- replaces \div division symbol. But we don't need that anyway...
\renewcommand{\div}{\ensuremath{\operatorname{div}}}

\newcommand{\rot}{\ensuremath{\operatorname{rot}}}
\newcommand{\curl}{\ensuremath{\operatorname{curl}}}

% custom command for transpose operator
\newcommand{\tr}{\ensuremath{\operatorname{tr}}}

\DeclareMathOperator{\sbullet}{\scalebox{0.65}{\ensuremath{\bullet}}}



% shortcuts for colored text
\newcommand{\cgn}[1]{{\color{green}#1}}
\newcommand{\crd}[1]{{\color{red}#1}}
\newcommand{\cbl}[1]{{\color{blue}#1}}
\newcommand{\cor}[1]{{\color{orange}#1}}
\newcommand{\cvt}[1]{{\color{violet}#1}}


\newcommand{\warn}[1]{
    \tikz[baseline=($(current bounding box.center)!0.65!(current bounding box.south)$)]{
        \node[isosceles triangle, 
              isosceles triangle stretches, 
              shape border rotate=90, 
              draw=red, 
              text=red,
              semithick,
              inner sep=0mm,
              minimum width=3mm, 
              minimum height=2.5mm, 
              rounded corners=0.5mm] {\bfseries \raisebox{0.1mm}{\footnotesize !}};}\crd{\textbf{ #1}}}

% \newcommand{\warnsymbol}{
%     \hspace{0.5mm}
%     \tikz[baseline=($(current bounding box.center)!0.65!(current bounding box.south)$)]{
%         \node[isosceles triangle, 
%               isosceles triangle stretches, 
%               shape border rotate=90, 
%               draw=red, 
%               text=red,
%               semithick,
%               inner sep=0mm,
%               minimum width=3mm, 
%               minimum height=2.5mm, 
%               rounded corners=0.5mm] {\bfseries \raisebox{0.1mm}{\footnotesize !}};}
% }


\newcommand{\txtqt}[1]{\textquotedbl #1\textquotedbl}

% bullet command for items in tables
\newcommand{\tabitem}{~~\llap{\textbullet}~~}


% customizes watermark and page color after a certain page limit
% colors all pages after the specified limit red
% source: https://stackoverflow.com/questions/2720534/force-a-maximum-number-of-pages-in-latex 
\newcounter{page@count}
\setcounter{page@count}{0}
\gdef\maxpages{\pagelimit}
\ifx\latex@outputpage\@undefined\relax% ChkTeX 21
    \global\let\latex@outputpage\@outputpage% ChkTeX 21
\fi%
\gdef\@outputpage{% ChkTeX 21
    \addtocounter{page@count}{1}%
    \ifnum\value{page@count}>\maxpages\relax%
        % change page background to red and add watermark
        \SetWatermarkText{\pagelimit\ Seiten Limit erreicht!}%
        \SetWatermarkScale{0.35}%
        \pagecolor{red}
        \latex@outputpage%
    \else%
        \SetWatermarkText{}%
        \latex@outputpage%
    \fi%
}


% remove title from table of contents, needed for layout
\renewcommand{\tableofcontents}{%
    \@starttoc{toc}
}


% scale super- and subscript -- not used currently, instead resized math font
% \catcode`_=\active% chktex 41 --> suppress ChkTeX warning
% \catcode`^=\active% chktex 41
% \newcommand_[1]{\ensuremath{\sb{\mathrm{\scaleobj{0.7}{#1}}}}}
% \newcommand^[1]{\ensuremath{\sp{\mathrm{\scaleobj{0.7}{#1}}}}}


\makeatother


% new environment for layout --> automatically adjusts to landscape or portrait
\NewDocumentEnvironment{layout}{}
                            {\ifthenelse{\paperwidth > \paperheight} % ChkTeX 1
                            % LANDSCAPE LAYOUT
                            {\begin{multicols*}{3} % ChkTeX 15
                                \begin{minipage}[t]{0.18\columnwidth}
                                    \vspace{-0.225\columnwidth}
                                    \qrcode[level=L, 
                                            version=0,
                                            height=0.9\columnwidth]{\repo}\\[1mm]
                                    \normalfont\footnotesize V \version{}
                                    \smallskip
                                \end{minipage}\hfill
                                \begin{minipage}[t]{0.81\columnwidth}
                                    \raggedright%
                                    \normalfont\huge\bfseries\title{}\\
                                    \normalfont\Large\semester\ -- \dozent{}\\
                                    \normalfont\large Autoren:\\
                                    \normalfont\large\author{}\\[2mm]
                                \end{minipage}
                                \normalfont\normalsize\myul{\url{\repo}}
                                \raggedcolumns}
                            % PORTRAIT LAYOUT
                            {\hfill\null\vspace{1cm}
                            \begin{center}
                                \normalfont\fontsize{35}{32}\selectfont\bfseries\title{}\\[7.5mm]
                                \normalfont\huge\semester\ \dozent{}\\
                                \normalfont\Large Autoren:\\
                                \normalfont\Large\author{}\\[2mm]
                                \normalfont\large Version:\\
                                \normalfont\large\version{}\\
                                \normalfont\normalsize\myul{\url{\repo}}\\[2mm]
                                \qrcode[level=L, 
                                        version=0,
                                        height=2cm]{\repo}
                            \end{center}
                            \vfill
                            \section*{\contentsname}
                            \begingroup
                            \setlength{\columnsep}{5mm}
                            \begin{multicols}{2}
                                \tableofcontents%
                            \end{multicols}
                            \endgroup
                            \vfill
                            \thispagestyle{empty}
                            \newpage
                            \begin{multicols*}{2}
                            \raggedcolumns}
                         }
                         {\end{multicols*}}


% new environment for centered tabulars
\NewDocumentEnvironment{ctabular}{m}
                        {\center\tabular{#1}}
                        {\endtabular\endcenter}


\NewDocumentEnvironment{rrcases}{}
                        {\left.\begin{aligned}}
                        {\end{aligned}\right\rbrace}
% ---------------------------------------- LISTINGS SETUP -----------------------------------------

% % hack to fix asterisk in lstlisting
% \makeatletter
% \lst@CCPutMacro%
%     \lst@ProcessOther{"2A}{% ChkTeX 18
%          {\raisebox{0.125pt}{*}}}
%     \@empty\z@\@empty% ChkTeX 21
% \makeatother


% % inline lst tikz node for later referencing
% \newcommand{\lstnode}[2][0.5ex]{
% 	\tikz[overlay, remember picture, inner sep=0pt, yshift=#1, minimum width=0mm]\node(#2){};
% }


% custom inline listings with box around them
% \newcommand{\mylstbox}[2][columns=fullflexible]{
%     \mybox{
%         \lstinline[#1]{#2}}}
% \newcommand{\mytclstbox}[2][columns=fullflexible]{
%     \mybox{
%         \lstinline[basicstyle=\sffamily\footnotesize\color{#1}, columns=fullflexible]{#2}}}


% % renew command for lstinputlisting with less vertical spacing
% \renewcommand{\lstinputlisting}[2][]{
%     \begingroup
%     \vspace{-0.6\abovedisplayskip}
%     \lst@setcatcodes%
%     \lst@inputlisting[#1]{#2}
%     \vspace{-0.6\abovedisplayskip}
% }


% % listings style (code)
% \lstdefinestyle{mystyle}{
%     backgroundcolor=\color{backcolour},   
%     commentstyle=\color{commentcolour},
%     keywordstyle=\bfseries\color{keywordcolour},
%     numberstyle=\tiny\color{numbercolour},
%     stringstyle=\color{stringcolour},
%     basicstyle=\sffamily,
%     breakatwhitespace=false,
%     breaklines=true,
%     captionpos=b,
%     keepspaces=true,
%     numbers=left,
%     numbersep=2pt,
%     showspaces=false,
%     showstringspaces=false,
%     showtabs=false,
%     tabsize=4,
% 	xleftmargin=1em,
% 	language=Java,
%     extendedchars=true,
%     inputencoding=cp1252,
% 	columns=[l]fullflexible	% see: https://tex.stackexchange.com/questions/99416/latex-source-code-listing-with-less-space-between-characters or manual
% }


% % custom lstinline command with custom colors
% \def\purplst{\begingroup\color{violet}}
% \def\greenlst{\begingroup\color{green}}
% \def\redlst{\begingroup\color{red}}
% \def\ywlst{\begingroup\color{orange}}
% \def\bluelst{\begingroup\color{blue}}
% \def\endlstcol{\endgroup}


% % basic lstinline style without colors
% \lstdefinestyle{basestyle}{
%     backgroundcolor=\color{backcolour},
%     keywordstyle=\bfseries,
%     numberstyle=\tiny\color{numbercolour},
%     basicstyle=\sffamily\footnotesize,
%     breakatwhitespace=false,     
%     breaklines=true,
%     captionpos=b,
%     keepspaces=true,
%     numbers=none,
%     numbersep=2pt,
%     showspaces=false,
%     showstringspaces=false,
%     showtabs=false,
%     tabsize=4,
%     xleftmargin=0em,
%     language=Java,
%     columns=flexible,	% see: https://tex.stackexchange.com/questions/99416/latex-source-code-listing-with-less-space-between-characters or manual
% }


% \lstset{
%     style=mystyle,
%     morekeywords={final, override, enum, var, List, Set, Map, String, Object},
%     moredelim=[il][\textcolor{orange}]{¦¦},
%     moredelim=[is][\textcolor{orange}]{&&}{&&},
%     moredelim=[is][\textcolor{violet}]{@}{@},
% }
