\section{Vektoranalysis}
% TODO: Skalarer Durchfluss (?)
% TODO: Spezialfälle (?)
% TODO: Vektor Längenelement (?)

% TODO: Vektorfelder
\subsection{Vektorfelder}
\begin{itemize}
    \item Jedem Punkt $P$ im Raum ist ein Vektor $\vec{V}$ zugeordnet
    \item Kann als $\vec{V}(\vec{r})$ geschrieben werden, wobei $\vec{r}$ ein Ortsvektor mit fixem Ursprung $\vec{0}$ ist
    % \item Kann auch als Gradient eines Skalarfeldes $\phi$ geschrieben werden: $\vec{V} = \nabla \phi$. % TODO: bei bedarf einfügen
\end{itemize}


% TODO: Divergenz (Volumenableitung) (Cartesisch; Nabla / del-Operator)
\subsection{Divergenz (Volumenableitung)}
\begin{outline}
    \1 Beschreibt, wie stark sich ein Vektorfeld in einem Punkt ausbreitet oder zusammenzieht
    \1 Beispiel: Vektorfeld das die Geschwindigkeit von Wasser in eineem Fluss beschreibt
        \2 An Punkten mit positiver Divergenz fliesst Wasser hinaus (Quelle)
        \2 An Punkten mit negativer Divergenz fliesst Wasser hinein (Senke)
\end{outline}

$\boxed{\nabla\vec{V} = \div \vec{V} = \lim_{\Delta V\to 0}\frac{\oint_{\scriptscriptstyle (S)}\vec{V}\cdot\diff\vec{S}}{\Delta V}}$


\subsubsection{Kartesisch}
\[
    \boxed{%
        \div \vec{V}
        = \nabla\cdot\vec{V}
        = \underbrace{%
            \left\lgroup%
                \frac{\partial}{\partial x};\frac{\partial}{\partial y};\frac{\partial}{\partial z}%
            \right\rgroup}_{\nabla} \cdot 
        \begin{pmatrix}
            V_x \\ V_y \\ V_z
        \end{pmatrix}
        = \frac{\partial V_x}{\partial x} + \frac{\partial V_y}{\partial y} + \frac{\partial V_z}{\partial z}
    }
\]


\subsubsection{Zylinderkoordinaten}
\[
    \div \vec{V} = \frac{1}{r} \frac{\partial}{\partial r} (rV_r) + \frac{1}{r} \frac{\partial V_\varphi}{\partial \varphi} + \frac{\partial V_z}{\partial z}
\]

% TODO: doublecheck this!!
% \subsubsection{Kugelkoordinaten}
% \[
%     \frac{1}{r^2}\frac{\partial}{\partial r} (r^2 V_r) + \frac{1}{r\sin\vartheta}\frac{\partial}{\partial \vartheta} (\sin\vartheta V_\vartheta) + \frac{1}{r\sin\vartheta}\frac{\partial V_\varphi}{\partial \varphi}
% \]



% TODO: Poisson-Gleichung (Laplace-Gleichung) -- two subsections?
\subsection{Poisson-Gleichung (Laplace-Gleichung)}


$\boxed{\Delta \phi
    = \div\left\lgroup\grad(\phi)\right\rgroup
    = \nabla^2 \phi
    = \frac{\partial^2 \phi}{\partial x^2} + \frac{\partial^2 \phi}{\partial y^2} + \frac{\partial^2 \phi}{\partial z^2}
    = f(\vec{r})}$
\begin{tabular}{O<{:} l}
    \Delta & Laplace-Operator\\
    \phi & Potentialfeld\\
    f(\vec{r}) & Quellfunktion
\end{tabular}

\subsubsection{Laplace-Gleichung}
$\boxed{\Delta \phi = f = 0}$ \textrightarrow\ Spezialfall der Poisson-Gleichung ohne äussere Quellfunktion

% TODO: Green'sche Funktion / Green'scher Satz
% TODO: Beispiel Poisson-Gleichung


% TODO: Rotation eines Vektorfelds / Rotationsfeld (rot(); curl)
\subsection[Rotation eines Vektorfelds (rot(), curl())]{Rotation eines Vektorfelds ($\rot()$, $\curl()$)}
\[
    \boxed{%
        \rot \vec{A}
        = \nabla \times \vec{A}
        =   \begin{pmatrix}
                \frac{\partial}{\partial x}\\
                \frac{\partial}{\partial y}\\
                \frac{\partial}{\partial z}
            \end{pmatrix} \times 
        \begin{pmatrix}
            A_x \\ A_y \\ A_z
        \end{pmatrix} =
        \frac{\partial A_x}{\partial x} + \frac{\partial A_y}{\partial y} + \frac{\partial A_z}{\partial z}
    }
\]

Gauss: $\div \rot(\vec{A}) \overset{!}{=} 0$

% TODO: Stokes Integralsatz % TODO: evtl als subsubsection?
\subsection{Stokes Integralsatz}


% TODO: Anwengungen: Maxwell-Gleichungen
\subsection{Anwendungen: Maxwell-Gleichungen}
% TODO: Koordinatensysteme (Kartesisch, Polar, Kugel (Geografisch & Math.))

