\section{Vektoranalysis}
% TODO: Skalarer Durchfluss (?)
% TODO: Spezialfälle (?)
% TODO: Vektor Längenelement (?)

% TODO: Vektorfelder
\subsection{Vektorfelder}
\begin{itemize}
    \item Jedem Punkt $P$ im Raum ist ein Vektor $\vec{V}$ zugeordnet
    \item Kann als $\vec{V}(\vec{r})$ geschrieben werden, wobei $\vec{r}$ ein Ortsvektor mit fixem Ursprung $\vec{0}$ ist
    % \item Kann auch als Gradient eines Skalarfeldes $\phi$ geschrieben werden: $\vec{V} = \nabla \phi$. % TODO: bei bedarf einfügen
\end{itemize}


% TODO: Divergenz (Volumenableitung) (Cartesisch; Nabla / del-Operator)
\subsection{Divergenz (Volumenableitung)}
\begin{outline}
    \1 Beschreibt, wie stark sich ein Vektorfeld in einem Punkt ausbreitet oder zusammenzieht
    \1 Beispiel: Vektorfeld das die Geschwindigkeit von Wasser in eineem Fluss beschreibt
        \2 An Punkten mit positiver Divergenz fliesst Wasser hinaus (Quelle)
        \2 An Punkten mit negativer Divergenz fliesst Wasser hinein (Senke)
\end{outline}

\[
    \boxed{\nabla\dotp\vec{V} = \div \vec{V} = \lim_{\Delta V\to 0}\frac{\oint_{\scriptscriptstyle (S)}\vec{V}\dotp\diff\vec{S}}{\Delta V}}
\]


\subsubsection{Kartesisch}
\[
    \boxed{%
        \div \vec{V}
        = \nabla\dotp\vec{V}
        = \underbrace{%
            \left\lgroup%
                \frac{\partial}{\partial x};\frac{\partial}{\partial y};\frac{\partial}{\partial z}%
            \right\rgroup}_{\nabla} \dotp 
        \begin{pmatrix}
            V_x \\ V_y \\ V_z
        \end{pmatrix}
        = \frac{\partial V_x}{\partial x} + \frac{\partial V_y}{\partial y} + \frac{\partial V_z}{\partial z}
    }
\]


\subsubsection{Zylinderkoordinaten}
\[
    \div \vec{V} = \frac{1}{r} \frac{\partial}{\partial r} (rV_r) + \frac{1}{r} \frac{\partial V_\varphi}{\partial \varphi} + \frac{\partial V_z}{\partial z}
\]

% TODO: doublecheck this!!
% \subsubsection{Kugelkoordinaten}
% \[
%     \frac{1}{r^2}\frac{\partial}{\partial r} (r^2 V_r) + \frac{1}{r\sin\vartheta}\frac{\partial}{\partial \vartheta} (\sin\vartheta V_\vartheta) + \frac{1}{r\sin\vartheta}\frac{\partial V_\varphi}{\partial \varphi}
% \]


% TODO: Integralsatz von Gauss
\subsection{Integralsatz von Gauss}
\[
    \boxed{\int\limits_{(V)} \div \vec{A} \diff V = \oint\limits_{(S) = \partial V} \vec{A} \dotp \diff \vec{S}}
\]
Fluss durch eingeschlossenen Körper = Gesamter Fluss durch geschlossenen Rand des Körpers

% TODO: Poisson-Gleichung (Laplace-Gleichung) -- two subsections?
\subsection{Poisson-Gleichung (Laplace-Gleichung)}


$\boxed{\Delta \phi
    = \div\left\lgroup\grad(\phi)\right\rgroup
    = \nabla^2 \phi
    = \frac{\partial^2 \phi}{\partial x^2} + \frac{\partial^2 \phi}{\partial y^2} + \frac{\partial^2 \phi}{\partial z^2}
    = f(\vec{r})}$
\begin{tabular}{O<{:} l}
    \Delta & Laplace-Operator\\
    \phi & Potentialfeld\\
    f(\vec{r}) & Quellfunktion
\end{tabular}

\subsubsection{Laplace-Gleichung}
$\boxed{\Delta \phi = f = 0}$ \textrightarrow\ Spezialfall der Poisson-Gleichung ohne äussere Quellfunktion

% TODO: Green'sche Funktion / Green'scher Satz
% TODO: Beispiel Poisson-Gleichung


% DONE: Rotation eines Vektorfelds / Rotationsfeld (rot(); curl)
\subsection[Rotation eines Vektorfelds (rot(), curl())]{Rotation eines Vektorfelds ($\rot()$, $\curl()$)}
Beschreibt, wie stark und in welche Richtung sich ein Vektorfeld an einem Punkt rotiert. Wobei der Vektor selbst die Rotationsachse beschreibt 
und dessen Betrag proportional zur Rotationsgeschwindigkeit ist.
Beispiel: Wirbelfelder
\[
    \boxed{%
        \rot \vec{A}
        = \nabla \times \vec{A}
        =   \begin{pmatrix}
                \frac{\partial}{\partial x}\\
                \frac{\partial}{\partial y}\\
                \frac{\partial}{\partial z}
            \end{pmatrix} \times 
        \begin{pmatrix}
            A_x \\ A_y \\ A_z
        \end{pmatrix} =
        \begin{pmatrix}
            \frac{\partial A_z}{\partial y} - \frac{\partial A_y}{\partial z}\\
            \frac{\partial A_x}{\partial z} - \frac{\partial A_z}{\partial x}\\
            \frac{\partial A_y}{\partial x} - \frac{\partial A_x}{\partial y}
        \end{pmatrix}
    }
\]

\begin{outline}
    \1 $\big|\rot \vec{A}\,\big| < 0$: Uhrzeigersinn
    \1 $\big|\rot \vec{A}\,\big| = 0$: Wirbelfrei
    \1 $\big|\rot \vec{A}\,\big| > 0$: Gegenuhrzeigersinn
\end{outline}

Gauss: $\div \left\lgroup\rot(\vec{A})\right\rgroup \overset{!}{=} 0$ % TODO: Gehört das hier hin?


% TODO: Integralsatz von Stokes % TODO: evtl als subsubsection?
\subsection{Integralsatz von Stokes}
\[
    \boxed{\oint\limits_{\scriptscriptstyle (C) = \partial S} \vec{A} \dotp \diff \vec{r} = \int\limits_{\scriptscriptstyle (S)} \rot \vec{A} \dotp \diff \vec{S}}
\]
$\partial S$ \textbf{muss} anhand Rechter-Hand-Regel orientiert sein.

Stokes sagt aus, dass die Summe der Verwirbelungen in einer Fläche, der Summe der Vektoren dessen Randes entsprechen.
% XXX: vector for evaluating purposes. Rest of vectors might be replaced by this
% Testvector: $\vec{V}\quad\tikz{\node[inner sep=0pt, minimum width=1pt, minimum height=1pt](a){$V$};\draw[line width=0.3pt]($(a.north west)!0.5!(a.north) + (0, 1pt)$) -- ($(a.north)!0.75!(a.north east)+(0,1pt)$) -- ++(-1pt,1pt);}$

% TODO: Anwengungen: Maxwell-Gleichungen
\subsection{Anwendungen: Maxwell-Gleichungen}
% TODO: Koordinatensysteme (Kartesisch, Polar, Kugel (Geografisch & Math.))

\crd{--TBD--}

