\section{Vektoranalysis}
% TODO: Skalarer Durchfluss (?)
% TODO: Spezialfälle (?)
% TODO: Vektor Längenelement (?)

% TODO: Vektorfelder
\subsection{Vektorfelder}
Das Vektorfeld \[\vec{V}:\mathbb{R}^n \rightarrow \mathbb{R}^n\] weist jedem Punkt $P \in \mathbb{R}^n$ einen Vektor $\vec{v} \in \mathbb{R}^n$ zu.
Die Notation eines Vektorfelds ist gleich deren eines Vektors, wobei Vektorfelder üblicherweise gross geschrieben werden.
Weiter kann auch $\vec{V}(\vec{x})$ geschrieben werden, wobei $\vec{x}$ der Stützvektor eines beliebigen Punktes ist.

\subsection{Gradient}
Wir erinnern uns an den Nabla- oder Del-Operator aus Kapitel \ref{section:diff_dgl_gradient_bivar:gradient} als Spaltenvektor der verschiedenen Raumableitungen:
\[
    \nabla 
    = \begin{pmatrix} \frac{\partial}{\partial x_1} & \frac{\partial}{\partial x_2} & \dots & \frac{\partial}{\partial x_n} \end{pmatrix}^T
\]
Der Gradient eines Potentialfelds $\phi: \mathbb{R}^n \to \mathbb{R}$ berechnet sich als 
\[
    \nabla \cdot \phi(\vec{x}) 
    = \begin{pmatrix} \frac{\partial}{\partial x_1} & \frac{\partial}{\partial x_2} & \dots & \frac{\partial}{\partial x_n} \end{pmatrix}^T \cdot \phi(\vec{x}) 
    = \begin{pmatrix} \frac{\partial \phi}{\partial x_1}(\vec{x}) & \frac{\partial \phi}{\partial x_2}(\vec{x}) & \dots & \frac{\partial \phi}{\partial x_n}(\vec{x}) \end{pmatrix}^T 
    = \vec{F}(\vec{x})
\]
und resultiert in einem Vektorfeld.
\begin{itemize}
    \item Wird als Potential das elektrische Potential verwendet, entspricht $\vec{F}$ dem (negativen, skalierten) elektrischen Feld.
    \item Wird als Potential eine Höhe verwendet, entspricht $\vec{F}$ der negativen Hangabtriebskraft.
    \item Der Gradient kann als mehrdimensionale Ableitung verstanden werden.
    \item Der Gradient steht senkrecht auf allen Kontouren unz zeigt in Richtung hoher wert.
    \item Die Multiplikation $\nabla \cdot \phi$ wird oft als $\nabla \phi$ abgekürzt.
    \item Zudem kann der Gradient auch als $\grad \phi$ geschrieben werden.
\end{itemize}

% TODO: Divergenz (Volumenableitung) (Cartesisch; Nabla / del-Operator)
\subsection{Divergenz}
Die Divergenz eines Vektorfelds
\[
\begin{aligned}
    \nabla \dotp \vec{V}(\vec{x})
    &= \begin{pmatrix} \frac{\partial}{\partial x_1} & \frac{\partial}{\partial x_2} & \dots & \frac{\partial}{\partial x_n} \end{pmatrix}^T \dotp \begin{pmatrix} v_1(\vec{x}) & v_2(\vec{x}) & \dots & v_n(\vec{x}) \end{pmatrix} \\
    &= \frac{\partial v_1}{\partial x_1}(\vec{x}) + \frac{\partial v_2}{\partial x_2}(\vec{x}) + \dots + \frac{\partial v_n}{\partial x_n}(\vec{x})
\end{aligned}
\]
ist ein Skalarfeld, das beschreibt, wie stark das Vektorfeld an einem gegebenen Punkt ``nach aussen gerichtet'' ist. 

\medskip
\begin{outline}
    \1 Wird als Vektorfeld die Fliessgeschwindigkeit einer Flüssigkeit eingesetzt, so entspricht die Divergenz dem Fluss aus einem Punkt heraus.
        \2 An Punkten mit positiver Divergenz fliesst Flüssigkeit hinaus (Quelle)
        \2 An Punkten mit negativer Divergenz fliesst Flüssigkeit hinein (Senke)
    \1 Wird das E-Feld eingesetzt, so entspricht die Divergenz der Ladungsdichte.
        \2 Pos. Ladungsdichte entspricht pos. Divergenz, bewirkt eine Quelle im E-Feld.
        \2 Neg. Ladungsdichte entspricht neg. Divergenz, bewirkt eine Senke im E-Feld.
    \1 Das Skalarprodukt $\nabla \dotp \vec{V}$ wird in der Regel ausgschreiben, um eine einfache Multiplikation handelt, wie das beim Gradienten der Fall ist.
    \1 Die Notation $ \div \vec{V}$ ist ebenfalls gebräuchlich.
\end{outline}

\medskip
Eine alternative und gut visualisierbare Definition der Divergenz, ist in zwei dimensionen
\[
    \div \vec{V} = \nabla\dotp\vec{V}  = \lim_{A \to 0}\frac{\oint_{C = \partial A} \vec{V}(\vec{x}) \dotp \hat{x} \diff \vec{x}}{A},
\]
wobei $A$ eine Fläche und $C$ dessen Kontur darstellt.

Verallgemeinert für die Anwendung in mehr als 2 Dimensionen lautet die Definitin
\[
    \nabla\dotp\vec{V} = \div \vec{V} = \lim_{\Omega \to 0}\frac{\oint_{C = \partial \Omega}\vec{V}(\vec{x}) \dotp \hat{x} \diff \vec{x}}{\Omega},
\]
wobei $\Omega$ ein Bereich im Raum $\mathbb{R}^n$ und C dessen Kontur ist.

\subsubsection{Verschiedene Koordinatensysteme}
% TODO: gerade kein bock 
\myul{\textbf{Kartesisch:}}
\[
    \boxed{
        \div \vec{V}
        = \nabla\dotp\vec{V}
        = \underbrace{%
            \left\lgroup%
                \frac{\partial}{\partial x};\frac{\partial}{\partial y};\frac{\partial}{\partial z}%
            \right\rgroup}_{\nabla} \dotp 
        \begin{pmatrix}
            V_x \\ V_y \\ V_z
        \end{pmatrix}
        = \frac{\partial V_x}{\partial x} + \frac{\partial V_y}{\partial y} + \frac{\partial V_z}{\partial z}
    }
\]

% TODO: gerade kein bock
\myul{\textbf{Zylinderkoordinaten:}}
\[
    \boxed{
        \div \vec{V} = \frac{1}{r} \frac{\partial}{\partial r} (rV_r) + \frac{1}{r} \frac{\partial V_\varphi}{\partial \varphi} + \frac{\partial V_z}{\partial z}
    }
\]

% TODO: doublecheck this!!
\myul{\textbf{Kugelkoordinaten:}}
% \[
%     \frac{1}{r^2}\frac{\partial}{\partial r} (r^2 V_r) + \frac{1}{r\sin\vartheta}\frac{\partial}{\partial \vartheta} (\sin\vartheta V_\vartheta) + \frac{1}{r\sin\vartheta}\frac{\partial V_\varphi}{\partial \varphi}
% \]

\subsection[Laplace Operator Delta]{Laplace Operator $\Delta$}
Der Laplaceoperator ist nichts anderes als die Divergenz des Gradienten eines Skalarfelds und vergleichbar mit der zweiten Ableitung.
Folglich gilt
\[
    \Delta \phi = \nabla \dotp (\nabla \phi) = \nabla^2 \phi = \frac{\partial^2 \varphi_1}{\partial x_1^2} + \frac{\partial^2 \varphi_2}{\partial x_2^2} + \dots + \frac{\partial^2 \varphi_n}{\partial x_n^2},
\]
wobei das Resultat ein Skalarfeld ist.

\subsection[Rotation eines Vektorfelds (rot, curl)]{Rotation eines Vektorfelds ($\rot$, $\curl$)}
Die Rotation eines Vektorfelds, auch Curl genannt, beschreibt, wie stark ein Vektorfeld um einen gegebenen Punkt ``rotiert'' und wird als

\begin{minipage}{0.8\linewidth}
    \[
        \rot \vec{V}
        = \nabla \times \vec{V}
        =   \begin{pmatrix}
                \frac{\partial}{\partial x}\\
                \frac{\partial}{\partial y}\\
                \frac{\partial}{\partial z}
            \end{pmatrix} \times 
        \begin{pmatrix}
            V_x \\ V_y \\ V_z
        \end{pmatrix} =
        \begin{pmatrix}
            \frac{\partial V_z}{\partial y} - \frac{\partial V_y}{\partial z}\\
            \frac{\partial V_x}{\partial z} - \frac{\partial V_z}{\partial x}\\
            \frac{\partial V_y}{\partial x} - \frac{\partial V_x}{\partial y}
        \end{pmatrix}
    \]
\end{minipage}
\hfill
\begin{minipage}{0.19\linewidth}
    \begin{center}
        \tdplotsetmaincoords{70}{110}
        \begin{tikzpicture}[tdplot_main_coords, scale=1]
            % Ebene
            \foreach \x in {-0.75,-0.5,...,0.75}
            {
                \draw[gray,very thin] (\x, -0.75, 0) -- (\x, 0.75, 0);
                \draw[gray,very thin] (-0.75, \x, 0) -- (0.75, \x, 0);
            }
            
            % Curl Vektor
            \draw [-{latex}] (0, 0, 0) -- (0, 0, 1) node [right] {$\nabla \times \vec{V}$};

            % Rechte Winkel
            \tdplotsetrotatedcoords{0}{90}{0}
            \draw[tdplot_rotated_coords, gray, very thin] (0, 0.15, 0) arc (90:180:0.15);
            \fill[tdplot_rotated_coords, gray] (-0.05, 0.04, 0) arc (-90:270:0.02);
            \tdplotsetrotatedcoords{-90}{90}{0}
            \draw[tdplot_rotated_coords, gray, very thin] (0, 0.15, 0) arc (90:180:0.15);
            \fill[tdplot_rotated_coords, gray] (-0.05, 0.04, 0) arc (-90:270:0.02);

            % Vektorfeld
            \draw [-{latex}] (0.5,0,0) arc (0:360:0.5) node [pos=0.7, above left, fill=white, rounded corners, inner sep=1pt] {$\vec{V}$};
        \end{tikzpicture}
    \end{center}
\end{minipage}
berechnet.
Der resultierende Vektor ist dabei die Rotationsachse, wobei die Rechte-Hand-Regel gilt.

Der Curl ist grundsätzlich nur in drei Raumdimensionen definiert.
Wenn die Rotation eines auf der Ebene $z=0$ definierten Vektorfelds berechnet werden soll, kann die obige Formel mit $V_z = 0$ angepasst werden:
\[
    \rot \vec{V}(x, y)
    = \nabla \times \vec{V}(x, y)
    =   \begin{pmatrix}
            \frac{\partial}{\partial x}\\
            \frac{\partial}{\partial y}\\
            \frac{\partial}{\partial z}
        \end{pmatrix} \times 
    \begin{pmatrix}
        V_x \\ V_y \\ 0
    \end{pmatrix} =
    \begin{pmatrix}
        0 \\
        0 \\
        \frac{\partial V_y}{\partial x} - \frac{\partial V_x}{\partial y}
    \end{pmatrix}
\]

\begin{outline}
    \1 Mit dem Curl-Operator kann z.B. elegant beschrieben werden, dass Wirbel im E-Feld auf zeitliche Änderungen im magnetischen Feld zurückzuführen sind:
        \2[] $\nabla \times \vec{E} = -\frac{\partial \vec{H}}{\partial t}$
\end{outline}

\subsection[Rechenregeln mit Nabla]{Rechenregeln mit $\nabla$}
TODO: KA, ob das alles stimmt, sieht auf den ersten blick jedoch plausiebel aus.

\myul{Gradient:}
\[ \nabla(\nabla \cdot \vec{A}) = \nabla \times \nabla \times \vec{A} + \nabla^2 \vec{A}\]
\[ \nabla(f \cdot g) = (\nabla f) \cdot g + f \cdot (\nabla g) \]
\[ \nabla(\vec{A} \cdot \vec{B}) = (\vec{A} \cdot \nabla) \vec{B} + (\vec{B} \cdot \nabla) \vec{A} + \vec{A} \times (\nabla \times \vec{B}) + \vec{B} \times (\nabla \times \vec{A}) \]
\myul{Divergenz:}
\[ \nabla \cdot (\nabla f) = \nabla^2 f \]
\[ \nabla \cdot (\nabla \times \vec{A}) = 0 \]
\[ \nabla \cdot (f \cdot \vec{A}) = (\nabla f) \cdot \vec{A} + f \cdot (\nabla \cdot \vec{A}) \]
\[ \nabla \cdot (\vec{A} \times \vec{B}) = (\nabla \times \vec{A}) \cdot \vec{B} - \vec{A} \cdot (\nabla \times \vec{B}) \]
\myul{Curl:}
\[ \nabla \times (\nabla f) = 0 \]
\[ \nabla \times (\nabla \times \vec{A}) = \nabla (\nabla \cdot \vec{A}) - \nabla^2 \vec{A} \]
\[ \nabla \times (\nabla^2 \vec{A}) = \nabla^2 (\nabla \times \vec{A}) \]
\[ \nabla \times (f \cdot \vec{A}) = (\nabla f) \times \vec{A} + f \cdot \nabla \times \vec{A} \]
\[ \nabla \times (\vec{A} \times \vec{B}) = (\vec{A} \cdot \nabla) \vec{A} - (\vec{A} \cdot \nabla) \vec{B} + \vec{A} \cdot (\nabla \cdot \vec{B}) - \vec{B} \cdot (\nabla \cdot \vec{A}) \]

\subsection{Anwendungen}

% TODO: Integralsatz von Gauss
\subsection{Integralsatz von Gauss}
\[
    \boxed{\int\limits_{(V)} \div \vec{A} \diff V = \oint\limits_{(S) = \partial V} \vec{A} \dotp \diff \vec{S}}
\]
Fluss durch eingeschlossenen Körper = Gesamter Fluss durch geschlossenen Rand des Körpers

% TODO: Poisson-Gleichung (Laplace-Gleichung) -- two subsections?
\subsection{Poisson-Gleichung (Laplace-Gleichung)}


$\boxed{\Delta \phi
    = \div\left\lgroup\grad(\phi)\right\rgroup
    = \nabla^2 \phi
    = \frac{\partial^2 \phi}{\partial x^2} + \frac{\partial^2 \phi}{\partial y^2} + \frac{\partial^2 \phi}{\partial z^2}
    = f(\vec{r})}$
\begin{tabular}{O<{:} l}
    \Delta & Laplace-Operator\\
    \phi & Potentialfeld\\
    f(\vec{r}) & Quellfunktion
\end{tabular}

\subsubsection{Laplace-Gleichung}
$\boxed{\Delta \phi = f = 0}$ \textrightarrow\ Spezialfall der Poisson-Gleichung ohne äussere Quellfunktion

% TODO: Green'sche Funktion / Green'scher Satz
% TODO: Beispiel Poisson-Gleichung


% DONE: Rotation eines Vektorfelds / Rotationsfeld (rot(); curl)


% TODO: Integralsatz von Stokes % TODO: evtl als subsubsection?
\subsection{Integralsatz von Stokes}
\[
    \boxed{\oint\limits_{\scriptscriptstyle (C) = \partial S} \vec{A} \dotp \diff \vec{r} = \int\limits_{\scriptscriptstyle (S)} \rot \vec{A} \dotp \diff \vec{S}}
\]
$\partial S$ \textbf{muss} anhand Rechter-Hand-Regel orientiert sein.

Stokes sagt aus, dass die Summe der Verwirbelungen in einer Fläche, der Summe der Vektoren dessen Randes entsprechen.
% XXX: vector for evaluating purposes. Rest of vectors might be replaced by this
% Testvector: $\vec{V}\quad\tikz{\node[inner sep=0pt, minimum width=1pt, minimum height=1pt](a){$V$};\draw[line width=0.3pt]($(a.north west)!0.5!(a.north) + (0, 1pt)$) -- ($(a.north)!0.75!(a.north east)+(0,1pt)$) -- ++(-1pt,1pt);}$

% TODO: Anwengungen: Maxwell-Gleichungen
\subsection{Anwendungen: Maxwell-Gleichungen}
% TODO: Koordinatensysteme (Kartesisch, Polar, Kugel (Geografisch & Math.))

\crd{--TBD--}

