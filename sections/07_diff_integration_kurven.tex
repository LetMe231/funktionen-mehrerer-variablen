\section{Differenziation und Integration von Kurven}
\subsection{Kurvenintegral 1. Art}
Mit dem Kurvenintegral 1. Art wird die Länge einer Kurve in einer Ebene oder im Raum bestimmt.

\subsubsection{Zweidimensional}
Um die Länge einer Kurve $K$, die durch $f(x, y)$ beschrieben wird, in der Ebene zu bestimmen, wird über das Linienelement $ds$ integriert:
\[
    \iint\limits_{K} ds = \int_{x_1}^{x_2} \int_{y_1}^{y_2} \sqrt{dx^2 + dy^2}
\]
Dabei ist es nötig, die Funktion $f(x)$ in die Parameterdarstellung $f(x(t), y(t))$ zu bringen, da der Ausdurck $\sqrt{dx^2 + dy^2}$ problematisch ist.
Es folgt
\[
    \int_{x_1}^{x_2} \int_{y_1}^{y_2} f(x, y) \sqrt{dx^2 + dy^2} \frac{dt}{dt} = \int_{t_0}^{T} \sqrt{(\frac{dx}{dt})^2 + (\frac{dy}{dt})^2} dt,
\]
wobei $\frac{dx}{dt}$ und $\frac{dy}{dt}$ Funktionen sind, die durch Ableiten von $x(t)$ bzw. $y(t)$ nach $t$ berechnet werden können.

\subsubsection{Dreidimensional}
Das Kurvenintegral 1. Art in drei Dimensionen wurde bereits in Kapitel \ref{section:int_multivar:länge_einer_fkt} beschrieben.
% TODO: evtl. hier hin verschieben

\subsection{Kurvenintegral 2. Art}
Beim Kurvenintegral 2. Art wird nicht die tatsächliche Länge einer Funktion, sondern die Länge deren Projektion auf eine Achse bestimmt.
Dazu wird stat über alle Koordinatenrichtungen nur über eine der Koordinaten integriert.

Es folgen einige Paare von Kurvenintegralen 2. Art entlang einer Kontur $K$ für Funktionen in expliziter Form und in Parameterdarstellung.

\myul{2D, Projektion auf x:}
\[
    \int\limits_{K}f(x, y)dx = \int_{t_0}^{T}f(x(t), y(t)) \cdot x\prime(t) \cdot dt
\]

\myul{3D, Projektion auf x:}
\[
    \int\limits_{K}f(x, y, z)dx = \int_{t_0}^{T}f(x(t), y(t), z(t)) \cdot x\prime(t) \cdot dt
\]

\subsubsection{Anwendungen}
TODO: Für was wird das gebraucht?!
