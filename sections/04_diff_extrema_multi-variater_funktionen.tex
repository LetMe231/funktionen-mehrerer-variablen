\newpage
\section{Extrema von Funktionen mehrerer Variabeln finden}

\begin{enumerate}[itemsep=1ex]
    \item \textbf{Gradient von $f$ Null-setzten und kritische Stellen finden:}
    
    $\nabla f=
    \begin{pmatrix}
        f_x\\
        f_y\\
        \vdots \\
        f_t
    \end{pmatrix} \stackrel{!}{=}
    \begin{pmatrix}
        0\\
        0\\
        \vdots \\
        0
    \end{pmatrix}
    \quad \Rightarrow x_0$, $y_0$, $\ldots$, $t_0$ bestimmen
    
    \item \textbf{Zweite Partielle Ableitungen für Hesse-Matrix H bestimmen:}
    
    \begin{minipage}[t]{0.48\columnwidth}
        $\mathbf{H}=\begin{pmatrix}
            f_{xx}&f_{xy}&\ldots &f_{xt}\\
            f_{yx}&f_{yy}&\ldots&f_{yt}\\
            \vdots &\vdots &\ddots &\vdots \\
            f_{tx}&f_{ty}&\ldots&f_{tt}
        \end{pmatrix}$
    \end{minipage}\hfill
    \begin{minipage}[c]{0.48\columnwidth}
        \begin{itemize}
            \item Symmetrien beachten!
            \item Nicht doppelt rechnen!
            \item[] \textrightarrow\ $f_{xt} = f_{tx}$
        \end{itemize}
    \end{minipage}

    \item \textbf{Hesse-Matrix H mit gefundenen Stellen füllen:}

    $\mathbf{H}(x_0,y_0,\ldots t_0)=
    \begin{pmatrix}
        f_{xx}(x_0,y_0,\ldots t_0)&f_{xy}(x_0,y_0, \ldots t_0)&\cdots &f_{xt}(x_0,y_0,\ldots t_0)\\
        f_{yx}(x_0,y_0,\ldots t_0)&f_{yy}(x_0,y_0, \ldots t_0)&\cdots &f_{yt}(x_0,y_0,\ldots t_0)\\
        \vdots &\vdots &\ddots &\vdots \\
        f_{tx}(x_0,y_0,\ldots t_0)&f_{ty}(x_0,y_0, \ldots t_0)&\cdots &f_{tt}(x_0,y_0,\ldots t_0)\end{pmatrix}$

    \item \textbf{Eigenwerte $\lambda_i$ der Hesse-Matrix bestimmen:}

    $\text{det}\left(\mathbf{H}(x_0,y_0,\ldots t_0) - \lambda \cdot \mathbf{E}\right)  = 0$

    Nullstellen $\lambda_i$ finden $\rightarrow  $ Eigenwerte

    \medskip
    \myul{Zur Erinnerung:}\\
    $\mathbf{E} = \begin{pmatrix}
        1&0&\ldots &0\\
        0&1&\ldots&0\\
        \vdots &\vdots &\ddots &\vdots \\
        0&0&\ldots&1\\
    \end{pmatrix}
    , \quad
    \lambda \cdot \mathbf{E} = \begin{pmatrix}
        \lambda&0&\ldots &0\\
        0&\lambda&\ldots&0\\
        \vdots &\vdots &\ddots &\vdots \\
        0&0&\ldots&\lambda\\
    \end{pmatrix}$

    \medskip
    $\mathbf{H}(x_0,y_0,\ldots t_0) - \lambda \cdot \mathbf{E} = \ldots \\
    \ldots  = 
    \begin{pmatrix}
        f_{xx}(x_0,y_0,\ldots t_0)- \lambda&f_{xy}(x_0,y_0, \ldots t_0)&\cdots &f_{xt}(x_0,y_0,\ldots t_0)\\
        f_{yx}(x_0,y_0,\ldots t_0)&f_{yy}(x_0,y_0, \ldots t_0)- \lambda&\cdots &f_{yt}(x_0,y_0,\ldots t_0)\\
        \vdots &\vdots &\ddots &\vdots \\
        f_{tx}(x_0,y_0,\ldots t_0)&f_{ty}(x_0,y_0, \ldots t_0)&\cdots &f_{tt}(x_0,y_0,\ldots t_0)- \lambda\end{pmatrix}$

    \item \textbf{Auswertung:}
    
    \begin{tabular}{lll}
        \hline
        $\lambda_i < 0 \,\,\,\forall i$ &$\Longrightarrow$& $\text{lokales Maximum}$\\
        \hline
        $\lambda_i > 0 \,\,\,\forall i$ &$\Longrightarrow$& $\text{lokales Minimum}$\\
        \hline
        $\lambda_i > 0\,$ und $\,\lambda_i < 0$ &$\Longrightarrow$& $\text{Sattelpunkt}$\\
        \hline
    \end{tabular}

    \medskip
    Erklärung:
    \begin{itemize}
        \item $\lambda_i < 0 \,\,\,\forall i$ $\Leftrightarrow $ Alle $\lambda_i$ sind negativ
        \item $\lambda_i > 0 \,\,\,\forall i$ $\Leftrightarrow $ Alle $\lambda_i$ sind positiv
    \end{itemize}
\end{enumerate}


\subsection{Lokales oder Globales Extremum}
Für eine beliebige die Funktion $f(x, y, \ldots  , t)$ gilt:

$\boxed{\begin{array}{llll}
    f(x,y,\ldots ,t)\leq M_{\max}&\forall(x,y,\ldots ,t)\in\mathbb{D}_f&\Rightarrow&\text{globales Maxinum}\\
    f(x,y,\ldots ,t)>M_{\max}&\exists(x,y,\ldots ,t)\in\mathbb{D}_f&\Rightarrow&\text{kein globales Maximum}\\
    \hline f(x,y,\ldots ,t)\geq M_{\min}&\forall(x,y,\ldots ,t)\in\mathbb{D}_f&\Rightarrow&\text{globales Minimum}\\
    f(x,y,\ldots ,t)<M_{\min}&\exists(x,y,\ldots ,t)\in\mathbb{D}_f&\Rightarrow&\text{kein globales Minimum}
\end{array}}$

\medskip
\begin{tabular}{ll}
    $M_{\max}$: &\text{grösstes lokales Maximum}\\
    $M_{\min}$: &\text{kleinstes lokales Minimum}
\end{tabular}

