\newpage
\section{Integration (bi-variat)}

\subsection{2D}

\[
    \int\int\limits_{\Omega}f(x;y)\cdot \diff x\cdot \diff y=\int\limits_{X}\left\lgroup\int\limits_{Y}f(x;y)\cdot dy \right\rgroup\cdot dx
\]

\[
    wenn\int\int|f(x;y)|dxdy<\infty
\]

\subsection{Normalbereich}

Schnitte sind Strecken (Intervalle) für x, y, ...

\subsection{Polar}

\[
    \text{d}x \cdot \text{d}y = r \cdot d\phi \cdot dr
\]

\subsection{2D Transformation Polar zu Kartesisch}
T $=$ Transformation
\[
    \text{Polar } (r,\phi) \xrightarrow{T} (x,y) \text{ Kartesisch}
\]

\[
\begin{pmatrix}
    x=r\cdot\cos(\varphi) \text{ } \cor{\mathbb{R}} \\
    y=r\cdot\sin(\varphi) \text{ } \cor{\mathbb{R}} 
\end{pmatrix}
\text{2D}
\]

Die Funktionen für $x$ und $y$ sind skalare Funktion.

    \begin{ctabular}{ll}
        $x=x(r;\varphi)$ & $ y=y(r;\varphi)$
    \end{ctabular}

\subsection{Derivative, Ableitung}


\subsection{3D Volumenberechnung}

$$V=\int_{x_{\min}}^{x_{\max}} \left[\int_{y_{\min}(x)}^{y_{\max}(x)} f\left(x;y\right)  \,dy \right] \,dx $$
\subsection{}
\subsection{}

