
\section{Anwendungen}

% TODO: Integralsatz von Gauss
\subsection{Integralsatz von Gauss}
\[
    \boxed{\int\limits_{(V)} \div \vec{A} \diff V = \oint\limits_{(S) = \partial V} \vec{A} \dotp \diff \vec{S}}
\]
Fluss durch eingeschlossenen Körper = Gesamter Fluss durch geschlossenen Rand des Körpers

% TODO: Poisson-Gleichung (Laplace-Gleichung) -- two subsections?
\subsection{Poisson-Gleichung (Laplace-Gleichung)}
$\boxed{\Delta \phi
    = \div\left\lgroup\grad(\phi)\right\rgroup
    = \nabla^2 \phi
    = \frac{\partial^2 \phi}{\partial x^2} + \frac{\partial^2 \phi}{\partial y^2} + \frac{\partial^2 \phi}{\partial z^2}
    = f(\vec{r})}$
\begin{tabular}{O<{:} l}
    \Delta & Laplace-Operator\\
    \phi & Potentialfeld\\
    f(\vec{r}) & Quellfunktion
\end{tabular}

\subsubsection{Laplace-Gleichung}
$\boxed{\Delta \phi = f = 0}$ \textrightarrow\ Spezialfall der Poisson-Gleichung ohne äussere Quellfunktion

% TODO: Green'sche Funktion / Green'scher Satz
% TODO: Beispiel Poisson-Gleichung
% TODO: Integralsatz von Stokes
\subsection{Integralsatz von Stokes}
\[
    \boxed{\oint\limits_{\scriptscriptstyle (C) = \partial S} \vec{A} \dotp \diff \vec{r} = \int\limits_{\scriptscriptstyle (S)} \rot \vec{A} \dotp \diff \vec{S}}
\]
$\partial S$ \textbf{muss} anhand Rechter-Hand-Regel orientiert sein.

Stokes sagt aus, dass die Summe der Verwirbelungen in einer Fläche, der Summe der Vektoren dessen Randes entsprechen.
% XXX: vector for evaluating purposes. Rest of vectors might be replaced by this
% Testvector: $\vec{V}\quad\tikz{\node[inner sep=0pt, minimum width=1pt, minimum height=1pt](a){$V$};\draw[line width=0.3pt]($(a.north west)!0.5!(a.north) + (0, 1pt)$) -- ($(a.north)!0.75!(a.north east)+(0,1pt)$) -- ++(-1pt,1pt);}$

% TODO: Anwengungen: Maxwell-Gleichungen
\subsection{Anwendungen: Maxwell-Gleichungen}
% TODO: Koordinatensysteme Geografisch

\crd{--TBD--}

