
\section{Anwendungen}

\subsection{Integralsatz von Gauss}
Der Integralsatz von Gauss 
\[
    \boxed{\oint\limits_{S = \partial V} \vec{A} \dotp \hat{n} \diff S = \int\limits_{V} \nabla \dotp \vec{A} \diff V}
\]
beschreibt, dass die aufintegrierte Divergenz in einem Körper gleich dem Fluss durch die Kontur dieses Körpers sein muss.
Die Normale $\hat{n}$ steht dabei senkrecht auf dem Oberflächenelement $\diff S$ und zeigt nach aussen.

\subsubsection{Green'sches Integraltheorem}
Das Green'sche Integraltheorem (auch Satz von Green) 
\[
    \boxed{\oint\limits_{C = \partial S} \vec{A} \dotp \hat{n} \diff l = \iint\limits_{S} \left(\frac{\partial A_y}{\partial x}-\frac{\partial A_x}{\partial y}\right) \diff x \diff y }
\]
ist der zweidimensionale Spezialfall des Integralsatzes von Gauss. Auch hier zeigt die normale $\hat{n}$ nach aussen.

\myul{\textbf{Green'sche Indentität Nr. 1}}
Wird $\vec{A} = U_1 \nabla U_2$ eingesetzt, so resultiert aufgrund der Produktregel
\[
    \oint\limits_{S = \partial V} (U_1 \nabla U_2) \dotp \hat{n} \diff S = \int\limits_{V} (U_1 \nabla U_2 + \nabla U_1 \dotp \nabla U_2) \diff V.
\]

\myul{\textbf{Green'sche Indentität Nr. 2}}
Wird $\vec{A} = U_1 \nabla U_2 - U_2 \nabla U_1 $ eingesetzt, so resultiert
\[
    \oint\limits_{S = \partial V} (U_1 \nabla U_2 - U_2 \nabla U_1) \dotp \hat{n} \diff S = \int\limits_{V} (U_1 \nabla U_2 - U_2 \nabla U_1) \diff V.
\]
Mit $U_1 = 1$ resultiert die etwas handlichere Indentität
\[
    \oint\limits_{S = \partial V} (\nabla U_2) \dotp \hat{n} \diff S = \int\limits_{V} (\nabla U_2) \diff V.
\]

\subsection{Integralsatz von Stokes}
Der Integralsatz von Stokes
\[
    \boxed{\int\limits_{S} \rot \vec{A} \dotp \hat{n} \diff S = \oint\limits_{C = \partial S} \vec{A} \dotp \diff \vec{r}}
\]
sagt aus, dass durch das Integrieren eines Vektorfelds $\vec{A}$ entlang der Kontur $C$ einer Fläche $S$ auf die mittleren Verwirbelungen im Innern der Fläche geschlossen werden kann.

Die Normale $\hat{n}$ und die Integrationsrichtung $\vec{r}$ müssen dabei die Rechte-Hand-Regel erfüllen.


\subsection{Poisson-Gleichung (Laplace-Gleichung)}
Die Poisson-Gleichung
\[
    \boxed{\Delta \phi (\vec{r})
    = f(\vec{r})}
    \hspace{1em}\text{oder}\hspace{1em}
    \boxed{\nabla^2 \phi (\vec{r})
    = f(\vec{r})}
\]
findet in der Physik oft Anwendung. $\phi$ beschreibt dabei ein skalares Potentialfeld, $f$ wird Quellenfunktion genannt und $\vec{r}$ ist ein beliebiger Stützvektor.

\subsubsection{Laplace-Gleichung}
Die Laplace-Gleichung
\[\boxed{\Delta \phi = f = 0}\] 
ist der Spezialfall der Poisson-Gleichung, bei dem keine Quellenfunktion $f$ besteht.


\subsection{Prinzip von d'Alambert}
Das Prinzip von d'Alembert ist ein Vorgehen zum Lösen von Wellengleichungen.
Die eindimensionale Wellengleichung
\[
    \frac{1}{c^2}\frac{\partial^2 u}{\partial t^2} - \frac{\partial^2 u}{\partial z^2}
\]
mit den Initialbedingungen
\[
    u(0, z) = f(z) 
    \hspace{1em}\text{bzw.}\hspace{1em}
    u(0, z) = f(z) \land \frac{\partial u}{\partial t} (0, z) = g(z)
\]
wird gelöst durch
\[
    u(t, z) = \frac{1}{2}(f(z + ct)+f(z-ct))
    \hspace{1em}\text{bzw.}\hspace{1em}
    u(t, z) = \frac{1}{2}(f(z + ct)+f(z-ct)) + \frac{1}{2c}\int_{z-ct}^{z+ct}g(s)ds.
\]


\subsection{Maxwell-Gleichungen}
\subsubsection{Gausssches Gesetz}
\[
    \nabla \cdot \vec{E} = \frac{\rho}{\varepsilon_0}
\]
\subsubsection{Gausssches Gesetz des Magnetismus}
\[
    \nabla \cdot \vec{B} = 0
\]
\subsubsection{Induktionsgesetz}
\[
    \nabla \times \vec{E} = -\frac{\partial \vec{B}}{\partial t}
\]
\subsubsection{Durchflutungsgesetz}
\[
    \nabla \times \vec{B} = \mu_0(\vec{J}+\varepsilon_0\frac{\partial \vec{E}}{\partial t})
\]
Zusammengesetzt aus dem \textbf{Ampèreschem Gesetz} $\nabla \times \vec{B} = \mu_0\vec{J}$ und Maxwells Erweiterung, der Verschiebungsstromdichte $\epsilon_0\frac{\partial \vec{E}}{\partial t}$.
