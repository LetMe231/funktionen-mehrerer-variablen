\section{Ableitungen, DGL und Gradienten}

\subsection{Partielle Ableitung}
Ableitung einer Partiellen Funktion. 

\example{Bi-Variate Funktion}
\[
    f(x, y):\, y \text{ fixieren} = \text{const.} = y_{0};\quad x \text{ \textbf{einzige} freie Variable}
\]

\subsubsection*{Notationen}

\begin{ctabular}{l l}
    1. Ordnung: & $\displaystyle f(x; y_{0})\Rightarrow \frac{\partial f}{\partial x} = f_{x}(x; y_{0})$ \\
    \multirow{2}*{2. Ordnung:}  &  $\displaystyle \frac{\partial}{\partial x}
                                    \left\lgroup\frac{\partial f}{\partial x}\right\rgroup = 
                                    \frac{\partial^{2}f}{\partial x^{2}} = f_{xx}$\\
                                &  $\displaystyle \frac{\partial}{\partial y}
                                    \left\lgroup\frac{\partial f}{\partial x}\right\rgroup = 
                                    \frac{\partial^{2}f}{\partial y \partial x} = f_{xy}$
\end{ctabular}


\subsubsection{Schwarz-Symmetrie}
Wenn \cor{$f_{xx}, f_{yy}, f_{xy}$ \& $f_{yx}$} \textbf{stetig} (sprungfrei) sind, dann gilt:
\[
    f_{xy} \overset{!}{=} f_{yx}
\]


\subsection{Gradient (Nabla-Operator)}
Spaltenvektor mit partiellen Ableitungen
\[
    \tikznode{grad}{\nabla} f = \begin{pmatrix}
        \frac{\partial f^{\mathstrut}}{\partial^{\mathstrut} x}\\
        \frac{\partial f^{\mathstrut}}{\partial^{\mathstrut} y}\\
        \vdots      % TODO: reformat vdots. They look like shit
    \end{pmatrix} \hat{=} \text{Vektorfeld}
\]

% TODO: some TikZ. Might not be needed and could be removed if unnecessary
\begin{tikzpicture}[overlay, remember picture, >={Latex}]
    \node[inner sep=0pt, ellipse, fit=(grad), draw, densely dotted]{};
    \node[below left= 2mm and 5mm of grad] (gtext) {\txtqt{Gradient}};
    \draw[->, rounded corners] (gtext.east) -| ($(grad.south)+(0,-1mm)$);
\end{tikzpicture}


\subsection{Totale Ableitung}
Für Fehlerrechnung benützt, da man hierbei die Abstände von $(x; y; z)$ 
zu einem festen Punkt $(x_{0}; y_{0}; z_{0})$ erhält. (relative Koordinaten)

\[
    D(f;\underbrace{(x_{0}, y_{0}, \ldots)}_{\text{Arbeitspunkt}}):\, 
    \rreal^{\tikznode{somea}{\scriptstyle 2}} \rightarrow \rreal^{\tikznode{someb}{\scriptstyle 1}};\, 
    \text{\txtqt{gute Approximation}}
\]
\[
    f(x = x_{0} + \Delta x; y = y_{0} + \Delta y; \ldots) = (D_{11}; D_{12})\cdot \begin{pmatrix}
        \Delta x\\
        \Delta y
    \end{pmatrix} + f(x_{0}; y_{0}) + \cgn{R_{1}}
\]
Wobei \cgn{$R_{1}$} dem \txtqt{Rest} entspricht. (Ähnlich wie bei Taylorreihe)

\begin{minipage}[c]{0.7\columnwidth}
    \[
        \frac{\cgn{R_{1}}}{\cor{d = \sqrt{\Delta x^{2} + \Delta y^{2}}}} \rightarrow 0 \text{ (\txtqt{gut}, \txtqt{schneller gegen 0 als $\cor{d}$})}
    \]
\end{minipage}\hfill
\begin{minipage}[c]{0.29\columnwidth}
    \begin{tikzpicture}[baseline=(current bounding box.center), 
                        >={Latex[width=1mm, 
                                 length=1mm]}, 
                        scale=0.5, 
                        font=\tiny]

        % Koordinatensystem
        \draw[->] (0,0) -- (4,0) node[below]{$x$};
        \draw[->] (0,0) -- (0,2.5) node[left]{$y$};

        % Punkte
        \node[circle, fill=black, inner sep=0pt, minimum size=2pt] (A) at (1.7,0.7) {};
        \node[circle, fill=black, inner sep=0pt, minimum size=2pt] (P) at (0.7,1.7) {};
        \node[inner sep=1pt, right=1mm of A] {$A=(x_{0}; y_{0})$};
        \node[inner sep=1pt, above right=0.1mm and -1ex of P] {$P=(x;y)$};
    
        % Differenz
        \draw (P) -- (A-|P) node[midway, left]{$\Delta y$} -- (A) node[midway, below]{$\Delta x$};
        \draw[orange] (P) -- (A) node[midway, above right, inner sep=1pt]{\cor{$d$}};
    \end{tikzpicture}
\end{minipage}

\[
    \begin{split}
        D(f;(x_{0}; y_{0})) &= \left\lgroup D_{11} = \frac{\partial f}{\partial x}(x_{0}; y_{0}); 
                                            D_{12} = \frac{\partial f}{\partial y}(x_{0}; y_{0})\right\rgroup\\
        &= (\nabla f)^{tr} \text{ \cbl{wenn $\frac{\partial f}{\partial x}; \frac{\partial f}{\partial y}$ stetig bei $A$}}
    \end{split}
\]

\begin{tikzpicture}[overlay, remember picture]
    \draw[tips, -{Latex}] (someb) to[out=135, in=45, edge node={node[above=0mm]{\tiny $1\times 2$ Matrix}}] (somea);
\end{tikzpicture}


\subsection{Linearapproximation (Tangentialapproximation)}
\[
    f(x;y) \approx f(x_{0}; y_{0}) + D(f;(x_{0}; y_{0}))\cdot \begin{pmatrix}
        \Delta x\\
        \Delta y
    \end{pmatrix}
    \quad\text{ linear in $\Delta x$ und $\Delta y$}
\]


\subsubsection{Tangentialebene}
\[
    \crd{g(x; y) = f(x_{0}; y_{0}) + D(f;(x_{0}; y_{0}))\cdot \begin{pmatrix}
        x - x_{0}\\
        y - y_{0}
    \end{pmatrix}}
\]


\subsubsection{Tangentialer Anstieg (Totale Differential)}
\[
    \cvt{\diff f \overset{!}{=} 
        \frac{\partial f}{\partial \tikznode{ptx}{x}}\diff x + 
        \frac{\partial f}{\partial \tikznode{pty}{y}}\diff y} 
        \quad \text{bezüglich } A=\underbrace{(x_{0}; y_{0})}_{\tikznode{axy}{}}
\]
\begin{tikzpicture}[overlay, 
                    remember picture, 
                    >={Latex[width=1mm, 
                             length=1mm]}]

    \draw[->] ($(axy)+(0, 1.8mm)$) to[bend left=10] (ptx.south east);
    \draw[->] ($(axy)+(0, 1.8mm)$) to[bend left=10] (pty.south east);
\end{tikzpicture}


\subsubsection{Differential-Trick (\texorpdfstring{$\diff f$}{df} Trick)}
\begin{minipage}[c]{0.5\columnwidth}
    \[\left\lgroup\begin{aligned}
        f &= c = \mathrm{const.} \quad | \diff(\ldots)\\
        \diff f &= \diff c \overset{!}{=} 0
    \end{aligned}\right\rgroup\]
\end{minipage}\hfill
\begin{minipage}[c]{0.5\columnwidth}
    \[
        f_{x}\diff x + f_{y}\diff y = 0 \quad \text{für Kontourlinien}
    \]
\end{minipage}


\subsubsection{Implizite (Steigungs-)Funktion}
\begin{minipage}[c]{0.6\columnwidth}
    \[
        \cbl{y'(x)} = \frac{\diff y}{\diff x} = -\frac{f_{x}}{f_{y}\crd{\neq 0}} \lor 
        \cbl{x'(y)} = \frac{\diff x}{\diff y} = -\frac{f_{y}}{f_{x}\crd{\neq 0}}
    \]
\end{minipage}\hfill
\begin{minipage}[c]{0.39\columnwidth}
    \begin{tikzpicture}[>={Latex[width=1mm,
                                 length=1mm]}, 
                        scale=0.75, 
                        font=\small]

        % Koordinatensystem
        \draw[->] (0,0) -- (3,0) node[below]{$x$};
        \draw[->] (0,0) -- (0,2) node[left]{$y$};
        
        % Funktionen
        \draw[color=green] plot[domain=0.55:1.5] (\x, {-((\x-1.5)*(\x-1.5))+1}) node[right]{$y$}; % y = -((x-1.5)^2)+1
        \draw[color=blue] node[above right]{$y'$} plot[domain=0.25:1.5] (\x, {\x-0.25}); % y = x-0.25

        % Arbeitspunkt
        \node[circle, fill=black, inner sep=0pt, minimum size=1.5pt] (P) at (1,0.75) {};
        \node[inner sep=0pt, above left=0.25mm and 0mm of P] {$P_{0}$};
        \draw[gray, dashed] (P) -- (P-|0,0) node[left]{$y_{0}$};
        \draw[gray, dashed] (P) -- (P|-0,0) node[below]{$x_{0}$};

        % Richtungselemente
        \draw[orange, semithick, ->] (P) -- (2,1.75) node[above, midway, rotate=45]{$\vec{r}$};
        \draw[->] (P) -- (P-|2,0) node[midway, below]{$\diff x$};
        \draw[->] (P-|2,0) -- (2,1.75) node[midway, right]{$\diff y$};
    \end{tikzpicture}
\end{minipage}


% \subsection{Differential}


\subsection{DGL}
\[
    y' = \tikznode{rhsfn}{\bbr{orange}{-\frac{f_{x}}{f_{y}}}};\, y(x_{0}) = y_{0}
\]
\tikz[overlay, remember picture] {\node[below=1mm of rhsfn, inner sep=0pt, font=\tiny, text=orange] {right-hand-side (r.h.s.) Funktion};}

\subsection{Richtungselement (Tangentiallinie an Kontouren)}


\subsection{Gradientenfeld \texorpdfstring{$\perp$}{\_|\_} Kontouren}
